\documentclass[diss]{template/setrem}

\usepackage[utf8]{inputenc}
\usepackage[table]{xcolor}
\usepackage{multicol}
\usepackage{array}
\usepackage{varwidth}
\usepackage{float}
\usepackage{todonotes}
\usepackage{subfigure}
\usepackage{graphicx,url}
\usepackage{lipsum}

\makeatletter
\g@addto@macro{\UrlBreaks}{\UrlOrds}
\makeatother

\usepackage{setspace}
\usepackage{amssymb}
\usepackage{colortbl}
\usepackage{color}
\usepackage{hyperref}
\usepackage{verbatim}
\usepackage{scrextend}
\usepackage{csvsimple}
\usepackage{glossaries}

\usepackage{listings}
\usepackage{xcolor}
\usepackage{threeparttablex}
\usepackage{lscape}
\usepackage{colortbl}
\usepackage{booktabs}

% Coloca a contagem das figuras sequenciais sem considerar capítulos
\usepackage{chngcntr}
\usepackage{tabularx}
\usepackage{amsmath}
\usepackage{datatool}
\usepackage{seqsplit}
\usepackage[toc,page]{appendix}

% The package option longtable redefines the \cline macro to work around a bug in longtable
\usepackage{longtable}
\usepackage{multirow}
\usepackage{titletoc}

\makeatletter
\def\@cline#1-#2\@nil{%
  \omit
  \@multicnt#1%
  \advance\@multispan\m@ne
  \ifnum\@multicnt=\@ne\@firstofone{&\omit}\fi
  \@multicnt#2%
  \advance\@multicnt-#1%
  \advance\@multispan\@ne
  \leaders\hrule\@height\arrayrulewidth\hfill
  \cr
  \noalign{\nobreak\vskip-\arrayrulewidth}}
\makeatother


\counterwithout{figure}{chapter}
\counterwithout{table}{chapter}

\definecolor{color_keywords}{rgb}{0.0, 0.0, 0.44}
\definecolor{color_mykeyword}{RGB}{10, 100, 112}

\definecolor{myblue}{rgb}{0,0.3,0.9}
\definecolor{mygreen}{rgb}{0,0.6,0}
\definecolor{mygray}{rgb}{0.5,0.5,0.5}
\definecolor{mymauve}{rgb}{0.58,0,0.82}
\definecolor{mygray2}{rgb}{0.2,0.2,0.2}
\definecolor{mygray3}{rgb}{0.4,0.4,0.4}



\lstdefinestyle{mycode}{ %
  %backgroundcolor=\color{yellow!05},   % choose the background color; you must add \usepackage{color} or \usepackage{xcolor}
  basicstyle=\linespread{1}\small,        % the size of the fonts that are used for the code \footnotesize
  %lineskip={-1pt},
  breakatwhitespace=false,         % sets if automatic breaks should only happen at whitespace
  breaklines=true,                 % sets automatic line breaking
  captionpos=t,                    % sets the caption-position to top
  commentstyle=\color{gray!90},    % comment style
%  deletekeywords={...},            % if you want to delete keywords from the given language
  escapeinside={\%*}{*)},          % if you want to add LaTeX within your code
  extendedchars=true,              % lets you use non-ASCII characters; for 8-bits encodings only, does not work with UTF-8
  frame=l,                    % adds a frame around the code (bt,l,r) single
  keepspaces=true,                 % keeps spaces in text, useful for keeping indentation of code (possibly needs columns=flexible)
  keywordstyle=\color{black}\bf,       % keyword style
  language=C++,                     % the language of the code
 % morekeywords={input,output,parallel_activity,stream_producer},            % if you want to add more keywords to the set
  keywordstyle=[2]{\color{blue}\bf},
 % keywords=[3]{serial_in_order, parallel, pipeline, run, add_filter, task_scheduler_init},
 % keywordstyle=[3]{\color{blue!70!green}\bf},
  numbers=left,                    % where to put the line-numbers; possible values are (none, left, right)
  numbersep=5pt,                   % how far the line-numbers are from the code
  numberstyle=\tiny\color{mygray}, % the style that is used for the line-numbers
  rulecolor=\color{black},         % if not set, the frame-color may be changed on line-breaks within not-black text (e.g. comments (green here))
  showspaces=false,                % show spaces everywhere adding particular underscores; it overrides 'showstringspaces'
  showstringspaces=false,          % underline spaces within strings only
  showtabs=false,                  % show tabs within strings adding particular underscores
  stepnumber=1,                    % the step between two line-numbers. If it's 1, each line will be numbered
  stringstyle=\color{red},     % string literal style
  tabsize=1,                       % sets default tabsize to 2 spaces
  title=\lstname                   % show the filename of files included with \lstinputlisting; also try caption instead of title
}

% Negrito no título do listing
\captionsetup[lstlisting]{font={bf},labelfont=bf}

%Ajusta a contagem do listing
\AtBeginDocument{% the counter is defined later
  \counterwithout{lstlisting}{chapter}%
}
\makeatletter
\renewcommand{\l@lstlisting}[2]{%
  \@dottedtocline{1}{0em}{1.5em}{\lstlistingname\ #1}{#2}%
}
\makeatother

\title{Desenvolvimento de um \emph{software} de repositório institucional de pesquisas preparado para ambientes baseados em nuvem}
\author{Alf}{Lucas Machado}
%optional
\advisor{Msc.}{Cesa Seibel}{Tiago}
% \advisor{Dr.}{LastName}{FirstName}
% \coadvisor{Dr.}{LastName}{FirstName}
% \coadvisor{Dr.}{LastName}{FirstName}

% Place where the undergraduate thesis will be made.
\location{Três de Maio}{RS}

% Date of undergraduate thesis development/presentation.
\date{Agosto}{2022}

% Course name - defined in setremdefs.sty.
\course{\ctrc}

% Header image of the cover.
\courseheader{\sistemasinformacaologo[1]}


\docname{Trabalho de Conclusão de Curso do Bacharelado em Sistemas de Informação - Faculdade Três de Maio - SETREM}

% Mostra a lista de tabelas e figuras com os prefixos corretos.
\titlecontents{table}
  [0em]
  {}
  {\tablename\enspace\thecontentslabel:\enspace\enspace}
  {}
  {\titlerule*[.5pc]{.}\contentspage}

\titlecontents{figure}
  [0em]
  {}
  {\figurename\enspace\thecontentslabel:\enspace\enspace}
  {}
  {\titlerule*[.5pc]{.}\contentspage}

\contentsuse{lstlisting}{lol}
\titlecontents{lstlisting}
    [0em]
    {}
    {\lstlistingname\enspace\thecontentslabel:\enspace\enspace}
    {}
    {\titlerule*[.5pc]{.}\contentspage}

\begin{document}


\maketitle
% \input{apro.tex}

% \begin{agradecimentos}
%   agradeço a todos
% \end{agradecimentos}

% \begin{dedicatoria}
%   dedico a todos
% \end{dedicatoria}

\keyword{Sistemas de Informação}
\keyword{Repositório Institutional}
\keyword{Computação em Nuvem}
\begin{resumo}
    \noindent

    % INTRO + TEMA
    Cada vez mais os repositórios institucionais estão sendo adotados por instituições de
    ensino superior, com o objetivo de reunir, preservar e oferecer o fácil acesso ao material
    científico produzido ao decorrer dos anos. Sendo por meio deste movimento de preservação
    digital da produção científica, que iniciativas de acesso aberto se tornam possíveis,
    possibilitando que uma produção científica escrita hoje possa ser vista e acessada
    de forma gratuita pelo mundo todo, podendo ser utilizada como base ou inspiração para
    diversas outras pesquisas. Tendo isto em mente, o tema desta pesquisa foi definido como
    o desenvolvimento de um repositório institucional de pesquisas acadêmicas baseado em nuvem.
    % OBJETIVO
    O objetivo deste trabalho consiste em desenvolver um repositório institucional
    de pesquisas acadêmicas projetado para ser executado em ambientes baseado em
    nuvem, utilizando de tecnologias como React e Vite para o frontend, Microsoft
    .NET 7.0 para a criação de APIs, PostgreSQL como banco de dados, e recursos de
    armazenamento de objetos baseados em nuvem.
    % PROBLEMA
    O problema desta pesquisa foi definido:
    Como projetar um \emph{software} de repositório institucional de pesquisas
    acadêmicas, utilizando de tecnologias de nível empresarial, e que esteja
    preparado para operar em ambientes baseados em nuvem?
    % METODOLOGIA - ABORDAGEM
    Como metodologia, foram utilizadas
    as abordagens dedutiva e quantitativa, a primeira abordagem foi utilizada de
    forma a utilizar o raciocínio lógico sobre pesquisas e repositórios acadêmicos
    já existentes, para melhor compreender o escopo da pesquisa, e assim propor soluções
    adequadas ao problema. Já a segunda abordagem foi utilizada para mensurar e analisar
    os dados numéricos referentes ao tempo de consultas e arquivamento das publicações no
    repositório acadêmico proposto.
    % METODOLOGIA - PROCEDIMENTOS
    Como procedimentos, foram utilizados a pesquisa
    bibliográfica e pesquisa ação, sendo a primeira utilizada na forma de de pesquisas
    em livros, artigos e revistas, com a finalidade de obter conhecimento sobre o
    tema pesquisado, e outros sistemas já existentes com propósitos semelhantes. Já o
    segundo procedimento foi utilizado na forma da pesquisa, aplicação e avaliação de
    tecnologias existentes, e que possam ser utilizadas durante o desenvolvimento do
    repositório acadêmico proposto.
    % METODOLOGIA - TÉCNICAS
    Como técnicas foram utilizadas as de prototipação
    de telas por meio de \emph{mockups} realizados com a ferramenta Figma, além das
    das técnicas de programação para o desenvolvimento do sistema, e de testes para
    verificar o funcionamento do \emph{software}.
    % RESULTADOS
    Como resultado, houve o desenvolvimento de um novo \emph{software web} de
    repositório institutional, que foi denominado de \emph{RESOAR} (\emph{Research Open Access Repository}),
    possuindo suporte a serviços de \emph{Object Storage} em nuvem como o \emph{Amazon S3} e \emph{Digital Ocean Spaces}.
    % CONCLUSÃO
    Como conclusão, o foco deste trabalho de desenvolver um novo repositório institucional
    de pesquisas acadêmicas, utilizando de tecnologias baseadas em nuvem, e unificar o
    processo de publicações e correções por parte dos alunos e orientadores, foi atingido.
    Proporcionando aos envolvidos um grande aprendizado nas diversas áreas que envolvem
    o desenvolvimento de \emph{software}, além de possibilitar a aplicação em prática dos
    conhecimentos adquiridos ao decorrer da graduação.

\end{resumo}


\keywordenglish{Information Systems}
\keywordenglish{Institutional Repository}
\keywordenglish{Cloud Computing}
\begin{abstract}
    % INTRO + TEMA
    \noindent
    \emph{Institutional repositories are increasingly being adopted by higher education
        institutions, with the aim of gathering, preserving and offering easy access to
        scientific material produced over the years. It is through this digital preservation
        movement of scientific production that open access initiatives become possible,
        enabling a scientific production written today to be seen and accessed free of
        charge throughout the world, and can be used as a basis or inspiration for several
        other researches. Bearing this in mind, the theme of this research was defined as
        the development of a cloud-based institutional repository of academic research.}
    % OBJETIVO
    \emph{The objective of this work is to develop an institutional repository of academic
        research designed to run in cloud-based environments, using technologies such as
        React and Vite for the frontend, Microsoft .NET 7.0 for creating APIs, PostgreSQL
        as a database, and cloud-based object storage capabilities.}
    % PROBLEMA
    \emph{The problem of this research was defined: How to design an institutional repository
        software for academic research, using enterprise-level technologies, and prepared to
        operate in cloud-based environments?}
    % METODOLOGIA - ABORDAGEM
    \emph{As methodology, deductive and quantitative approaches were used, the first approach
        was used in order to use logical reasoning on existing research and academic repositories,
        to better understand the scope of the research, and thus propose adequate solutions to the
        problem. The second approach was used to measure and analyze numerical data referring to the
        query time and archiving of publications in the proposed academic repository.}
    % METODOLOGIA - PROCEDIMENTOS
    \emph{As procedures, bibliographic research and action research were used, the first being
        used in the form of research in books, articles and magazines, in order to obtain
        knowledge about the researched topic, and other existing systems with similar purposes.
        The second procedure was used in the form of research, application and evaluation of
        existing technologies, that can be used during the development of the proposed
        academic repository.}
    % METODOLOGIA - TÉCNICAS
    \emph{As techniques, screen prototyping was used through mockups made with Figma tool,
        in addition to programming techniques for system development, and tests to verify
        the software's operation.}
    % RESULTADOS
    \emph{As a result, a new institutional repository web software was developed, which was
        called RESOAR (Research Open Access Repository), supporting cloud object storage
        services such as Amazon S3 and Digital Ocean Spaces.}
    % CONCLUSÃO
    \emph{In conclusion, the focus of this work of developing a new institutional repository
        of academic research, using cloud-based technologies, and unifying the process of
        publications and corrections by students and advisors, was achieved. Providing those
        involved with a great deal of learning in the various areas that involve software
        development, in addition to enabling put in practice the knowledge
        acquired during graduation.}

\end{abstract}


\begin{singlespaced}
    \listoffigures
\end{singlespaced}

\begin{singlespaced}
    \listoftables
\end{singlespaced}

% \renewcommand*{\lstlistlistingname}{LISTA DE CÓDIGOS}
% \lstlistoflistings

% \listofmyequations

\begin{listofabbrv}{OR-AC-GAN} % Put the largest abbreviation.
    \setstretch{1}
    \item[APCs] {\emph{Article Processing Charges}}
    \item[API] {\emph{Application Programming Interface}}
    \item[BBM] {Biblioteca Brasiliana Guita e José Mindlin}
    \item[BDTC] {Biblioteca Digital de Trabalhos Científicos}
    \item[BOIA] {\emph{Budapest Open Access Initiative}}
    \item[BPMN] {\emph{Business Process Model and Notation}}
    \item[BSD] {\emph{Berkeley Software Distribution}}
    \item[FINEP] {Financiadora de Estudos e Projetos}
    \item[Fiocruz] {Fundação Oswaldo Cruz}
    \item[GPL] {\emph{GNU General Public License}}
    \item[HMR] {Hot Module Replacement}
    \item[HTML] {\emph{HyperText Markup Language}}
    \item[IBICT] {Instituto Brasileiro de Informação em Ciência e Tecnologia}
    \item[IR] {\emph{Institutional Repository}}
    \item[ISO] {\emph{International Organization for Standardization}}
    \item[NBR] {Norma Brasileira}
    \item[OA] {\emph{Open Access}}
    \item[OAIS] {\emph{Open Archival Information System}}
    \item[OAR] {\emph{Open Access Repositories}}
    \item[OpenDOAR] {\emph{Directory of Open Access Repositories}}
    \item[RDF] {\emph{Resource Description Framework }}
    \item[RESOAR] {\emph{Research Open Access Repository}}
    \item[ROAR] {\emph{Registry of Open Access Repositories}}
    \item[ROARMAP] {\emph{Registry of Open Access Repositories Mandatory Archiving Policies}}
    \item[SETREM] {Sociedade Educacional Três de Maio}
    \item[URI] {\emph{Uniform Resource Identifier}}
    \item[USP] {Universidade de São Paulo}
\end{listofabbrv}

% 
\tableofcontents

\pagestyle{empty}
{
    \chapter*{Introdução} \label{chap:intro}

Os repositórios acadêmicos dentro de instituições de ensino
podem ser vistos como uma ferramenta essencial para a preservação
e propagação do conhecimento, visto que tais repositórios reúnem
o acervo das pesquisas realizadas durante passagem dos acadêmicos
pela instituição, e em teoria disponibilizam fácil acesso a este
material.

Todavia, conforme \cite{PORTO:difusao_cientifica_recortes}
muitas instituições de ciência e tecnologia acabam por não
dispor de uma estrutura profissionalizada de comunicação,
suporte e divulgação dos materiais produzidos, não contemplando
a comunicação em seu organograma funcional, e recorrendo a
improvisações na hora da disponibilização de meios de acesso
a divulgação de seus projetos.

Tendo isto em mente, o tema desta pesquisa surgiu como uma
continuação direta a um projeto de prática profissional
desenvolvido pelo próprio autor, durante o sétimo
semestre da graduação em Sistemas de Informação, na
Sociedade Educacional Três de Maio - SETREM, que possuía
como objetivo analisar o processo de armazenamento e
acesso as publicações realizadas dentro da instituição.

Durante a realização da prática profissional, foi constatado
que a faculdade não possuía um sistema repositório acadêmico
que possibilitasse aos alunos realizar a publicação de seus
artigos e pesquisas. Também foi verificado por meio da aplicação
de um questionário, que mais da metade dos acadêmicos
(cerca de 62,1\% das respostas) apresentavam interesse máximo
em uma escala de 1 a 5, pela utilização de um repositório acadêmico
institucional.

Ao pesquisar sobre repositórios acadêmicos existentes em plataformas
agregadoras como o OpenDOAR\footnote{https://v2.sherpa.ac.uk/opendoar/},
foi constatado que a maioria dos repositórios nacionais utilizam como
base o DSpace, um \emph{software open source} utilizado para a
criação de repositórios acadêmicos institucionais.

Este \emph{software} foi cogitado como um dos principais candidatos
para uma possível implantação dentro da instituição, porem ao verificar
trabalhos relacionados como o de \cite{2019:RodrigoMoreira},
foi relatado que por se tratar de um \emph{software open source},
fica a critério das instituição todas as questões pertinentes a
atualização de versões, personalização, segurança, privacidade dos
dados e \emph{backups} dos arquivos, oque resulta em um ambiente
fragmentado onde algumas instituições utilizam versões antigas
do DSpace, que por exemplo não funcionam de forma adequada em
dispositivos moveis, conforme relatado no trabalho relacionado
de \cite{2019:FernandesMacedes}.

Possuindo como inspiração os tópicos anteriormente citados,
este trabalho de conclusão de curso possui como objetivo a utilização
de tecnologias baseadas em nuvem para realizar a análise e
desenvolvimento de uma nova plataforma de repositório acadêmico,
que possibilite a interação entre os alunos e orientadores,
englobando o processo de correção e envio de revisões
de publicações dentro da própria plataforma, e que seja uma alternativa
as atuais plataformas de repositórios existentes.

Com a realização deste projeto de pesquisa, buscou-se expor a
importância da utilização de repositórios acadêmicos para realizar
a preservação digital dos materiais produzido, evitando assim a perda
ou esquecimento da produção acadêmica realizada dentro da instituição.
Além de trazer contribuição tanto para os acadêmicos da faculdade SETREM, quanto para outras
instituições que possam optar pela utilização do repositório
acadêmico desenvolvido.

Este trabalho foi estruturado da seguinte forma, no capítulo 1
é abordado o plano de estudo, elencando tópicos como a delimitação
do tema da pesquisa, objetivos, justificativa, metodologia, dentre outros.
No capitulo 2 é apresentado o embasamento teórico e conceitual
visto como necessário para realização deste trabalho.

No capítulo 3 são apresentados os resultados obtidos com a realização desta
pesquisa, protótipos e apresentação do repositório acadêmico proposto.
Em ultima parte, é apresentado a conclusão da pesquisa,
detalhando os objetivos alcançados, validação das hipóteses, dificuldades
e limitações encontradas durante o desenvolvimento, e propostas futuras
de pesquisa.
}
\chapter{Plano de Estudo e Pesquisa} \label{chap:ResearchPlan}

\section{Tema} \label{sec::Theme}
Desenvolvimento de um \emph{software} de repositório institucional de
pesquisas preparado para ambientes baseados em nuvem.

\subsection{Delimitação do Tema} \label{subsec::ThemeDelimitation}

A delimitação do tema se dará como o desenvolvimento de uma aplicação
web de repositório institucional, que envolva o processo de envio
de correções e revisões das publicações dentro da própria plataforma,
e que seja preparado desde a sua concepção para ser executado em ambientes
baseados em nuvem.

Para a criação do repositório institucional será utilizando de tecnologias
como React e Vite para o frontend, Microsoft .NET 6.0 para a criação
de APIs, PostgreSQL como banco de dados, e recursos de armazenamento baseados em
nuvem como o Amazon S3 ou Digital Ocean Spaces para o armazenamento de arquivos.

O desenvolvimento deste projeto de pesquisa foi realizado durante o período de
maio a dezembro de 2022, pelo acadêmico Lucas Machado Alf do curso bacharelado
em Sistema de Informação na Sociedade Educacional Três de Maio – SETREM.

\section{Objetivo Geral} \label{sec:objective}

Desenvolver um repositório institucional de pesquisas acadêmicas
baseado em nuvem, visando reunir e preservar as publicações acadêmicas
e científicas produzidas em âmbito universitário, além de unificar o processo
de publicações e correções por parte dos orientadores em uma única plataforma.

\subsection{Objetivos Específicos}
\begin{enumerate}
    \item Explorar ferramentas existentes de repositório acadêmico e pesquisar sobre trabalhos e artigos relacionados.
    \item Pesquisar sobre formas de armazenamento de documentos em nuvem e mecanismos de busca e análise textual de documentos.
    \item Elicitar os requisitos para o desenvolvimento do repositório acadêmico e definir as tecnologias que serão utilizadas para o desenvolvimento (Ex: banco de dados, linguagens e frameworks).
    \item Desenvolver uma aplicação web de repositório acadêmico, e popular o banco de dados com um \emph{dataset} de publicações acadêmicas.
    \item Realizar testes no repositório acadêmico desenvolvido, verificando o desempenho da aplicação em relação ao crescimento da quantidade de registros publicados na plataforma.
\end{enumerate}

\section{Justificativa}\label{sec:justification}

O tema desta pesquisa surgiu como uma continuação direta a um projeto
de Prática Profissional desenvolvido pelo próprio autor, durante o sétimo
semestre da graduação em Sistemas de Informação, na Sociedade Educacional
Três de Maio – SETREM, que tinha como objetivo analisar o atual processo
de armazenamento e acesso as publicações produzidas dentro da faculdade,
e realizar a representação gráfica dos processos por meio de diagramas BPMN
(\emph{Business Process Model and Notation}).

Conforme o próprio autor, durante o desenvolvimento da pesquisa foi constatado que a faculdade
não possuía um sistema de repositório institucional que possibilitasse aos
acadêmicos realizar a publicação de seus artigos, práticas profissionais,
interdisciplinares, TCCs e demais pesquisas realizadas na instituição.
Também foi verificado por meio da aplicação de um questionário, que
mais da metade dos acadêmicos (cerca de 62,1\% das respostas) apresentavam
interesse máximo em uma escala de 1 a 5, por um repositório institucional acadêmico.

Ao pesquisar sobre repositórios acadêmicos existentes em plataformas
agregadoras como o OpenDOAR\footnote{https://v2.sherpa.ac.uk/opendoar/},
foi constatado que a maioria dos repositórios nacionais utilizam como
base o DSpace, um \emph{software open source} escrito em Java para
criação de repositórios institucionais, sendo minoria as instituições
que utilizam um sistema de repositório institucional diferente deste.

No trabalho relacionado de \cite{2019:RodrigoMoreira} é possível
verificar que por se tratar de um \emph{software open source},
fica a critério das instituições as questões pertinentes a personalização
do DSpace, segurança e privacidade dos dados, \emph{backup} dos arquivos,
e a realização de atualização de versão, o que resulta em um ambiente
fragmentado onde algumas instituições utilizam versões antigas do \emph{software},
que por exemplo não funcionam de forma adequada em dispositivos móveis como
relatado no trabalho relacionado de \cite{2019:FernandesMacedes}.

Possuindo como inspiração os tópicos anteriormente citados,
este trabalho de conclusão de curso possui como objetivo e principal
contribuição cientifica, a utilização de tecnologias baseadas em
nuvem para realizar a análise e desenvolvimento de uma nova plataforma
de repositório acadêmico, que possibilite a interação entre os alunos
e orientadores, englobando o processo de correção e envio de revisões
de publicações dentro da própria plataforma, e que seja uma alternativa
as atuais plataformas de repositórios existentes.

Com a realização desta pesquisa, buscou-se trazer contribuição
tanto para os acadêmicos da faculdade SETREM, quanto para outras
instituições que possam optar pela utilização do repositório
acadêmico desenvolvido. Também houve o propósito de expor a
importância da utilização de repositórios acadêmicos para realizar
a preservação digital dos materiais produzido, evitando assim a perda
ou esquecimento da produção acadêmica e cientifica realizada dentro
da instituição.

% Com o aprendizado adquirido durante o desenvolvimento deste trabalho
% de conclusão de curso, foi possível realizar o aprofundamento do
% conhecimento nas áreas de tecnologia da informação, nuvem computacional,
% repositórios acadêmicos, gestão de conhecimento, linguagens de programação
% e demais tecnologias utilizadas durante o desenvolvimento da pesquisa.

\section{Problema} \label{sec::Problem}

Para a escolha do problema desta pesquisa foi ressaltado como sendo
o principal fator motivador, a necessidade de reunir as publicações acadêmicas
produzidas dentro de instituições em um repositório institucional,
que forneça rápido acesso às publicações científicas produzidas dentro da
instituição, tanto de forma interna para os estudantes da instituição,
quanto de forma aberta ao publico em geral.

Também foram levados em conta os fatores relatadas nos trabalhos
relacionados referentes a personalização, compatibilidade com
dispositivos moveis e dificuldades na atualização entre versões
dos \emph{softwares open source} de repositório institucional existes.

Sendo consideras tais premissas, o problema desta pesquisa foi definido:
Como projetar um \emph{software} de repositório institucional de pesquisas
acadêmicas, utilizando de tecnologias de nível empresarial, e que esteja
preparado para operar em ambientes baseados em nuvem?

\section{Hipóteses} \label{sec::Hypothesis}
\begin{enumerate}

    \item É possível manter um ambiente de infraestrutura em nuvem
          contendo os serviços de \emph{backend}, \emph{frontend}, banco de dados
          e armazenamento das publicações com um orçamento máximo de R\$ 500,00
          por mês.

    \item O recurso de Full Text Search presente no banco de dados PostgreSQL
          pode ser utilizado como uma alternativa viável para realizar
          as consultas por publicações dentro do repositório acadêmico
          proposto, (menos de 5 segundos por consulta), mesmo em bases de
          dados com mais de 30 mil publicações.

    \item O processo desenvolvido para extração do texto das publicações
          durante o auto arquivamento é rápido o suficiente para não
          necessitar de processamento em segundo plano (menos de 5 segundos),
          mesmo em publicações com mais de 200 páginas.

\end{enumerate}

\section{Metodologia} \label{sec:Methodology}

Conforme \citep[p. 15]{LOVATO:metodologia} a metodologia de pesquisa
consiste em um ramo da filosofia da ciência, que estuda os métodos que
os cientistas podem utilizar para chegar nos resultados de seus estudos.
Em outras palavras, tem como objetivo estudar os métodos que visam
conduzir um aumento do conhecimento sobre o tema pesquisado, preocupando-se
com o raciocínio, procedimentos e técnicas que podem ser utilizadas para
dar credibilidade aos resultados.

\subsection{Abordagem}

Os métodos de abordagem segundo \citep[p. 29]{LOVATO:metodologia} podem ser
divididos em dois grupos, o primeiro consiste no tipo de raciocínio que
é utilizado para se chegar aos resultados e conclusões. Já o segundo
se relaciona com a utilização, ou não, de análise numérica e estatística.

Segundo o mesmo autor, a abordagem quantitativa é utilizada quando
as conclusões são obtidas por meio de um conjunto de dados numéricos
e analise estatística, busca compreender melhor situações onde é possível
estabelecer relação entre variáveis, correlações ou causa-efeito.

Já segundo \cite{GIL:metodologia} a abordagem dedutiva, conforme a definição clássica,
é o método em que parte-se de um conceito geral, onde parte dos
conceitos são conhecidos como verdadeiros ou falsos, e se chega
a uma conclusão particular, por meio da aplicação da lógica.

Para o desenvolvimento deste trabalho, foram utilizadas as abordagens
dedutiva e quantitativa, a primeira abordagem foi utilizada
de forma a utilizar o raciocínio lógico sobre pesquisas e repositórios
acadêmicos já existes, para melhor compreender o escopo da pesquisa,
e assim propor soluções adequadas ao problema.

Já a segunda abordagem foi utilizada para mensurar e analisar
os dados numéricos referentes ao tempo de consultas e arquivamento
das publicações no repositório acadêmico proposto, além dos
custos envolvendo o armazenamento das publicações em nuvem.

\subsection{Procedimentos}

São apresentados neste tópicos os procedimentos utilizados
durante a realização deste trabalho, como a pesquisa bibliográfica e
o método de pesquisa ação.

\subsubsection{Pesquisa Bibliográfica}

Conforme \citep[p. 183]{LAKATOS2003:metodologia}, a pesquisa bibliográfica
abrange toda a bibliografia que se refere ao tema da pesquisa,
incluindo desde publicações avulsas, boletins, jornais, revistas,
livros, monografias, até meios de comunicação via rádio,
gravações e filmes, tendo como finalidade realizar o contanto
entre o pesquisador e o material já existente sobre o tema pesquisado.

Este procedimento foi utilizado por meio da realização de pesquisas
em livros, artigos e revistas, com a finalidade de se obter
conhecimentos sobre o tema pesquisado, e outros sistemas
já existentes com propósitos semelhantes.

\subsubsection{Pesquisa Ação}

Como elicitado por \citep[p. 45]{LOVATO:metodologia}, a pesquisa ação
diferente de outras abordagens em que o principal
embasamento consiste na literatura, o problema é real, e não existe
um plano previamente definido e inflexível, mas sim um refinamento
constante entre planejar, agir avaliar e refletir.

Por se tratar do desenvolvimento de um \emph{software}, o procedimento
de pesquisa ação foi utilizado na forma da pesquisa,
aplicação e avaliação de tecnologias existentes, e que possam
ser utilizadas durante o desenvolvimento do repositório
acadêmico proposto.

\subsection{Validação das Hipóteses}

Durante a elaboração do presente trabalho, foram elencadas
três hipóteses, sendo a  primeira "O recurso de Full Text
Search do banco de dados PostgreSQL pode ser utilizado como
uma alternativa viável para realizar as consultas por publicações
dentro do repositório acadêmico (menos de 5 segundos por consulta),
mesmo em bases de dados com mais de 30 mil publicações".

Para realizar a validação desta hipótese será utilizado do
arXiv Dataset\footnote{https://www.kaggle.com/Cornell-University/arxiv},
um dataset gratuito com mais de 1.7 milhões de artigos e publicações acadêmicas,
para popular a base de dados do repositório institucional. Como o arXiv Dataset
possui um grande número de registros, a base de dados será limitada
a cerca de 35 mil publicações.

Após realizar a preparação da base de dados, serão desenvolvidas
5 consultas diferentes, que serão executadas 30 vezes cada, de forma a obter um tempo
médio de execução por consulta. Todos os testes realizados serão executados utilizando
a versão 14 do PostgreSQL, em ambiente Windows 11, com as seguintes configurações de hardware:
Intel Core i3 9100, 16GB RAM DDR4 2400Mhz, 1TB SSD NVME WD Blue SN550.

A segunda hipótese é apresentada como "O processo desenvolvido
para extração do texto das publicações para pesquisa e indexação
é rápido o suficiente para não necessitar de processamento em
segundo plano (menos de 5 segundos), mesmo em publicações com mais
de 200 páginas". Esta segunda hipótese pode ser validada por meio
de um teste de desempenho, que consiste realizar a publicação de uma
mesma pesquisa com mais de 200 páginas diversas vezes, e mensurar
o tempo médio para conclusão do processo.

Já a terceira hipótese, consiste em "É possível manter um ambiente
de infraestrutura em nuvem contendo os serviços de backend,
frontend, banco de dados e armazenamento das publicações,
com um orçamento máximo de R\$ 500,00 por mês". Esta hipótese
pode ser validada por meio da comparação dos preços de diferentes
provedores de nuvem, com base nos requisitos mínimos estipulados
para o repositório acadêmicos proposto.

\subsection{Técnicas}

Conforme \citep[p. 174]{LAKATOS2003:metodologia} as técnicas podem
ser definidas como um conjunto de procedimentos que servem a
uma ciência ou arte, sendo utilizados para trazer tais
conceitos a parte prática, tendo em mente que toda ciência utiliza
de incontáveis técnicas para a obtenção de seus resultados.

Durante o desenvolvimento desta pesquisa, por se tratar do
desenvolvimento de um \emph{software}, foi utilizado da técnica
de prototipação de telas por meio de \emph{mockups} realizados
com a ferramenta Figma.

Em tradução livre de \cite{uzayr:mockups} um \emph{software mockup}
consiste em um desenho de uma página ou aplicação web,
que é desenvolvida para trazer vida a uma ideia, permitindo
que um designer possa examinar como diferentes elementos visuais
trabalham juntos. Os \emph{mockups} também permitem que as partes
interessadas ou \emph{Stakeholders} do projeto possam verificar como a
interface irá parecer, enquanto sugerem mudanças
apropriadas em relação a cores, imagens e estilos.

Também foram utilizadas as técnicas de análise estatística
e comparativa para realizar a analise de preços em diferentes
ambientes de nuvem, e a técnica de testes para verificar
o funcionamento do \emph{software} desenvolvido.

\section{Orçamento} \label{sec:budget}

De acordo com \citep[p. 128]{LAKATOS2021:metodologia} o orçamento
tem como objetivo responder a questão "quanto será necessário
despender?", distribuindo os gastos com o projeto em vários items.

\begin{table}[H]
    \caption{Orçamento}
    \begin{tabular}{|p{6.4cm}|c|r|r|}
        \hline
        {\textbf{Item}}                         & {\textbf{Quantidade}} & {\textbf{Valor Unitário}} & {\textbf{Valor Total}} \\ \hline
        {{Capa}}                                & {2}                   & {R\$ 25,00 }              & {R\$ 50,00}            \\ \hline
        {{Espirais}}                            & {3}                   & {R\$ 5,00 }               & {R\$ 15,00}            \\ \hline
        {{Horas trabalhadas}}                   & {800}                 & {R\$ 30,00 }              & {R\$ 24,000}           \\ \hline
        {{Impressões}}                          & {1000}                & {R\$ 0,30 }               & {R\$ 300,00}           \\ \hline
        {{Serviços de armazenamento em nuvem}}  & {1}                   & {R\$ 30,00 }              & {R\$ 30,00}            \\ \hline
        {{Serviços de Banco de Dados em nuvem}} & {1}                   & {R\$ 300,00 }             & {R\$ 300,00}           \\ \hline
        {{Serviços de Designer}}                & {1}                   & {R\$ 350,00 }             & {R\$ 350,00}           \\ \hline
        {{Serviços de hospedagem em nuvem}}     & {2}                   & {R\$ 100,00 }             & {R\$ 200,00}           \\ \hline
    \end{tabular}
\end{table}

\section{Cronograma de Atividades} \label{sec:schedule_activities_table}

Em conformidade com \citep[p. 128]{LAKATOS2021:metodologia} o cronograma
visa responder a pergunta “quando?”, sendo elaborado de forma a
representar a previsão do tempo necessário para realização da pesquisa.

\begin{table}[H]
    \caption{Cronograma}
    \begin{tabular}{|p{7cm}|l|l|l|l|l|l|l|}
        \hline
        \multicolumn{1}{|c|}{}                                      & \multicolumn{7}{c|}{\textbf{2022}}                                                                                                                                                                   \\ \cline{2-8}
        \multicolumn{1}{|c|}{\multirow{-2}{*}{\textbf{Atividades}}} & \textbf{Mai}                       & \textbf{Jun}             & \textbf{Jul}             & \textbf{Ago}             & \textbf{Set}             & \textbf{Out}             & \textbf{Nov}             \\ \hline
        Desenvolvimento do projeto                                  & \cellcolor[HTML]{000000}           & \cellcolor[HTML]{000000} &                          &                          &                          &                          &                          \\ \hline
        Entrega e validação do projeto                              &                                    & \cellcolor[HTML]{000000} &                          &                          &                          &                          &                          \\ \hline
        Desenvolvimento de requisitos funcionais                    &                                    & \cellcolor[HTML]{000000} &                          &                          &                          &                          &                          \\ \hline
        Desenvolvimento de protótipos                               &                                    & \cellcolor[HTML]{000000} & \cellcolor[HTML]{000000} &                          &                          &                          &                          \\ \hline
        Desenvolvimento do repositório acadêmico                    &                                    &                          & \cellcolor[HTML]{000000} & \cellcolor[HTML]{000000} & \cellcolor[HTML]{000000} & \cellcolor[HTML]{000000} &                          \\ \hline
        Organização de Dataset de trabalhos acadêmicos              &                                    &                          &                          &                          & \cellcolor[HTML]{C0C0C0} & \cellcolor[HTML]{C0C0C0} &                          \\ \hline
        Realização de testes no repositório acadêmico               &                                    &                          &                          &                          &                          & \cellcolor[HTML]{C0C0C0} & \cellcolor[HTML]{C0C0C0} \\ \hline
        Documentação dos resultados                                 &                                    &                          &                          &                          &                          &                          & \cellcolor[HTML]{C0C0C0} \\ \hline
        Entrega do Relatório Final                                  &                                    &                          &                          &                          &                          &                          & \cellcolor[HTML]{C0C0C0} \\ \hline
        Apresentação do Relatório Final                             &                                    &                          &                          &                          &                          &                          & \cellcolor[HTML]{C0C0C0} \\ \hline
    \end{tabular}
\end{table}

\begin{table}[H]
    \begin{tabular}{|
            >{\columncolor[HTML]{C0C0C0}}l |l|}
        \hline
                                                        & Proposto  \\ \hline
        \cellcolor[HTML]{000000}{\color[HTML]{000000} } & Realizado \\ \hline
    \end{tabular}
\end{table}
% Chapter 2

\chapter{Fundamentação Teórica}\label{chap:background}


\section{Área dos Negócios}\label{sec:business}


\section{Fundamentos da Área}\label{sec:fundamental}

Equação ~\ref{eq:my_equation} é um exemplo de uma equação em Latex:


\begin{equation}\label{eq:my_equation}
    h_t = f(W^{(hh)}h_{t-1} + W^{(hx)}x_t).
\end{equation}


Figura\ref{fig:diagram} é um exemplo para incluir uma figura.

\begin{figure}[htb]
    \caption{Simple diagram}
    \centering
    \includegraphics[scale=1.9]{img/diagram.pdf}
    \label{fig:diagram}
    \source{Retirado de \cite{larcc}.}
\end{figure}

\cite{GRIEBLER:IJPP:18}


\cite{MACCOOL:structured_patterns:book:12}


\section{Trabalhos Relacionados}\label{sec:rw}

São apresentados neste tópico, os trabalhos relacionados, que possuem
algum ponto em comum com o que se quer realizar no presente projeto.
Pretendesse aqui, identificar estes pontos, e compará-los com a proposta deste projeto.

\subsection{BDTC - Uma Biblioteca Digital de Trabalhos Científicos com Serviços Integrados}

O trabalho desenvolvido por \cite{CERVI:bdtc}, possui como premissa a apresentação
de uma proposta de biblioteca digital de trabalhos científicos, que foi denominada como
BDTC, que possui como objetivo prover o suporte a três pontos fundamentais:
auto-arquivamento do conteúdo, extração de metadados e busca de similaridade.

Como um dos principais diferenciais, foi desenvolvido um mecanismo de
busca por similaridade para as pesquisas realizadas na BDTC,
que permite ao usuário encontrar trabalhos relacionados, mesmo que
parte da palavra pesquisada não seja exatamente igual ao conteúdo presente
no documento.

Para o desenvolvimento deste mecanismo de busca, foi utilizado
um recurso denominado \emph{n-gram}, que permite quebrar uma palavra em
conjuntos de letras de tamanhos variados, é retratado como exemplo o
\emph{3-gram} do termo "Juci", que pode ser quebrado em dois conjuntos de 3
letras, ou seja, \emph{"Juc"} e \emph{"uci"}.

Estes conjuntos são armazenados em uma tabela de indices no banco de
dados, de forma que ao realizar uma consulta a partir de uma palavra,
o sistema retorna todos os trabalhos que contenham algum dos conjuntos
de letras que compõem a palavra pesquisada.

Em relação com o presente projeto, pode ser realizado uma comparação com
a forma como o mecanismo de busca foi desenvolvido na BDTC. Ao envés de
utilizar o recurso de \emph{n-gram}, o sistema proposto neste projeto utilizara
o recurso de \emph{tokenization} presente na ferramenta de Full Text Search do
banco de dados PostgreSQL, que possui um resultado final semelhante, porem não
idêntico, visto que o \emph{tokenization} remove o gênero das palavras,
os espaços em branco, palavras comuns e palavras que não são consideradas relevantes.

\subsection{Desenvolvimento da nova Biblioteca Digital da Biblioteca Brasiliana USP: Relato de Experiência}

O relato de experiência desenvolvido por \cite{GarciaRodrigoMoreira2019DdnB}
apresenta o desenvolvimento da na nova plataforma de Biblioteca Digital da
BBM, a Biblioteca Brasiliana Guita e José Mindlin, em forma de retrospectiva
desde o projeto-piloto, relatando os principais problemas, êxitos
e desafios encontrados durante o desenvolvimento do projeto, que envolvia
a digitalização e desenvolvimento de uma coleção digital para a biblioteca.

A Biblioteca Brasiliana Guita e José Mindlin foi inaugurada em março
de 2013, sendo um órgão e entidade acadêmica da Pró-Reitoria de Cultura
e Extensão da USP (Universidade de São Paulo). Este biblioteca envolve o
projeto Brasiliana USP, que foi iniciado em 2005, e tem como objetivo
abrigar a coleção Brasiliana, doada por José Mindlin. Dentro do escopo
deste projeto, em 2008 é iniciado o projeto-piloto da Biblioteca
Brasiliana Digital, que visa a preservação do acervo e democratização
do acesso ao material.

Para o desenvolvimento da biblioteca digital, foi optado por realizar uma
customização do sistema DSpace, um software open source de repositório
digital, com recursos como o Djatoka (servidor de imagens) e visualizadores
de livros como IIPImage e BookReader.

Como principais problemas e êxitos, foi ressaltado a forma como as customizações
foram realizadas no DSpace, sendo muitas delas realizadas diretamente no código fonte
do programa, tornando extramente difícil ou até impossibilitando a atualização
para novas versões da plataforma. Resultado em inconsistências na
visualização dos documentos digitalizados, lentidão do sistema, dentro outros
problemas que vieram a surgir ao longo do tempo.

Além disto, também foi relatado a rotativada das equipes como um fator de impacto
para a continuidade do desenvolvimento da biblioteca, que em sua
maioria era constituída por bolsistas, estagiários e poucos profissionais
terceirizados contratados por tempo determinado. Também foi constatado
que as máquinas digitalizadoras adquiridas para a biblioteca não eram
adequadas para o manuseio dos documentos do acervo, visto que os documentos
eram obras raras, e que necessitavam de diversos cuidados para a preservação
e conversação do material.

Com o tempo parte dos problemas foram resolvidos, sendo até adquirido novas
maquinas digitalizadoras, mais modernas e adequadas para o manuseio do material
bibliográfico.

Comparando o trabalho relacionado com o presente projeto de pesquisa,
é possível ressaltar que o atual projeto não tem como intenção a digitalização
de um acervo físico, porem o sistema DSpace que foi utilizado no trabalho
relacionado foi identificado como um sistema muito popular para o desenvolvimento
de repositórios acadêmicos e bibliotecas digitais de universidades, sendo
possível utilizar a experiência adquirida na implantação desse sistema durante
o desenvolvimento do repositório acadêmico proposto.

\subsection{Classificação facetada: proposta de categorias fundamentais para organizar teses e dissertações em uma biblioteca digital}

O artigo desenvolvido por \cite{PereiraClaytonMartins2021Cfpd} tem como
proposta a apresentação de categorias fundamentais, baseadas nos trabalhos
do matemático e bibliotecário indiano S. R. Ranganathan e do \emph{Classification Research Group}, que pode ser utilizadas
durante o desenvolvimento de interfaces de navegação facetada de bibliotecas
digitais e repositórios de dissertações e testes.

Neste trabalho é relatado que é comum em bibliotecas digitais de teses
e dissertações, encontrar problemas referente a facilidade de encontrar
e recuperar documentos neles armazenados, visto que a principal forma de
pesquisa tende a ser por um campo textual, onde é possível efetuar buscas
simples, e em alguns casos utilizar a combinação de operadores lógicos, como
\emph{AND}, \emph{OR} e \emph{NOT}, para buscas mais complexas.

Porem esta forma de pesquisa, por meio de um único campo textual,
exige que o usuário possua conhecimento prévio de noções de logica,
sigla dos cursos, areas e linhas de pesquisa, e que seja capaz de
utilizar essas informações para construir uma busca mais complexa,
geralmente fazendo com que as pesquisas realizadas retornem resultados
vazios, ou não exibam o total potencial de documentos contidos na plataforma,
levando em consideração questões semânticas dos termos utilizados na pequisa.

Desta forma, a abordagem de pequisa facetada permite que o usuário navegue
pela estrutura conceitual das informações armazenadas no repositório, além de
combinar conceitos de diferentes perspectivas ou facetas (janelas ou menus), sendo uma abordagem
de pesquisa mais eficiente, auxiliando o usuário a encontrar o que procura,
de forma visual e intuitiva a partir de palavras chaves modificáveis em um
vocabulário controlado.

Como resultado de sua pesquisa, foram obtidos as seguintes categorias
fundamentais, seguidas de um exemplo: Documento (Trabalho de conclusão), Tipo (Dissertação, Tese);
Curso (Meteorologia); Linha de Pequisa (Sensoriamento);
Tema (Anomalias climáticas); Especialização do tema (Conservação de Energia);
Localização (Oceano Atlântico); e Ano de publicação (2020).

Em relação ao atual projeto de pesquisa, as categorias fundamentais encontradas
no trabalho relacionado poderiam ser utilizadas para a organização dos documentos dentro do repositório
de trabalhos acadêmicos proposto, podendo ser utilizados em recursos de
filtragem das publicações dentro da plataforma.
% Chapter 3
\chapter{Resultados}\label{chap:Results}

Neste capítulo são apresentados os resultados do trabalho desenvolvido,
apresentando a criação do repositório institucional proposto, desde a
sua concepção, \emph{backlog} do produto, protótipos, resultado final
e validação das hipóteses.

\section{Concepção do Repositório Institucional}

A concepção do repositório institucional proposto foi realizada com base
nos desafios e problemas relatados nos trabalhos relacionados, e na análise
dos pontos positivos e negativos encontrados em outros sistemas de repositório
institucional existentes.

Como um dos principais diferenciais do repositório proposto, é que desde o
inicio de seu desenvolvimento ele foi projetado para ser facilmente implantado
e executado em ambientes baseados em nuvem, sendo assim todas as suas configurações
podem ser realizadas por meio de variáveis de ambiente, que em geral podem ser
facilmente definidas pelo próprio painel ou \emph{dashboard} fornecido pelo
provedor de nuvem. A ideia é nunca precisar acessar remotamente a máquina
em que a aplicação está sendo executada para realizar qualquer tipo de configuração.

Outro ponto de diferença é que o repositório proposto somente aceita arquivos no
formato PDF, visando mitigar problemas relacionados a formatação do documento, e
buscando garantir que o usuário ao acessar as publicações, visualize
o documento da mesma forma que o autor originalmente a escreveu. Esta redução
do escopo de formatos de arquivos aceitos pelo repositório, também garante que
todas as publicações estejam normalizadas em um mesmo formato, permitindo a
utilização de técnicas mais especializadas para a extração de texto dos documentos.

Para realizar a escolha do nome e identidade visual do repositório, foram buscados
por termos que remetessem a "Repositório", "Acesso Aberto'' e "Pesquisa", chegando
a escolha do nome RESOAR, uma sigla em inglês para \emph{Research Open Access Repository}.
A sigla "OAR'' que remete a \emph{Open Access Repository} já é baste difundida, e
pode ser encontrada em nomes como o OpenDOAR\footnote{https://v2.sherpa.ac.uk/opendoar/} e
ROARMAP\footnote{https://roarmap.eprints.org/}.

\begin{figure}[H]
    \caption{Identidade visual do repositório}
    \centering
    \frame{\includegraphics[scale=0.0558]{img/resoar.png}}
    \label{fig:resoar}
    \source{Do próprio autor}
\end{figure}

A Figura \ref{fig:resoar} apresenta a logo ou identidade visual do repositório,
que foi elaborada por um designer contratado da região de Três de Maio - RS,
que desenvolveu tanto a identidade visual quanto os primeiros protótipos do repositório.

Em sua composição é possível perceber um simbolo de lupa embutido na primeira letra "R''
representando a pesquisa. A cor padrão da logo foi escolhida como o laranja com
o código HEX \#ff5400, porém o simbolo também pode ser exibido em outras variações,
como em preto e branco, escala de cinza ou outras cores.

\section{Backlog do produto}

Nesta seção será apresentado o \emph{backlog} do produto que foi desenvolvido
seguindo a metodologia de \emph{User Stories}, contendo uma lista de funcionalidades
que o sistema final deve conter, além de \emph{mockups} das telas, servindo como uma
base para o desenvolvimento da aplicação.

\begin{table}[H]
    \caption{Tela de Login}
    \begin{tabular}{|p{1cm}|p{14cm}|}
        \hline
        \multicolumn{1}{|c|}{\textbf{01}} & \textbf{Tela de Login}                                                                                                                                                                                                                                                                        \\ \hline
        \multicolumn{2}{|l|}{\begin{tabular}[c]{@{}l@{}}\textbf{Como} usuário do sistema\\ \textbf{Eu quero} realizar o login utilizando meu email e senha\\ \textbf{Para que} possa utilizar as funções presentes no sistema.\end{tabular}}                                                                                              \\ \hline
        \multicolumn{2}{|l|}{\textbf{Critérios de aceitação}}                                                                                                                                                                                                                                                                             \\ \hline
        \multicolumn{2}{|l|}{\begin{tabular}[c]{@{}l@{}}1. Deve possui o botão de exibir/esconder a senha.\\2. Deve possuir os atalhos para as telas de recuperação de senha e cadastro\\ de usuário.\\3. Caso o usuário digitar a senha incorretamente mais de 3 vezes, deve solicitar\\ a solução de um desafio captcha. \end{tabular}} \\ \hline
        \multicolumn{2}{|c|}{\includegraphics[scale=0.296]{img/resoar-login.png}}                                                                                                                                                                                                                                                         \\ \hline
    \end{tabular}
\end{table}

\begin{table}[H]
    \caption{Cadastro de usuário}
    \begin{tabular}{|p{1cm}|p{14cm}|}
        \hline
        \multicolumn{1}{|c|}{\textbf{02}} & \textbf{Cadastro de usuário}                                                                                                                                                                                 \\ \hline
        \multicolumn{2}{|l|}{\begin{tabular}[c]{@{}l@{}}\textbf{Como} novo usuário\\ \textbf{Eu quero} me cadastrar no sistema\\ \textbf{Para que} possa entrar no sistema.\end{tabular}}                                                                \\ \hline
        \multicolumn{2}{|l|}{\textbf{Critérios de aceitação}}                                                                                                                                                                                            \\ \hline
        \multicolumn{2}{|l|}{\begin{tabular}[c]{@{}l@{}}1. Deve validar se já existe um usuário cadastrado com o mesmo email.\\2. Deve validar se os dois campos de senha são iguais.\\3. Deve solicitar a solução de um desafio captcha. \end{tabular}} \\ \hline
        \multicolumn{2}{|c|}{\includegraphics[scale=0.294]{img/resoar-new-account.png}}                                                                                                                                                                  \\ \hline
    \end{tabular}
\end{table}

\begin{table}[H]
    \caption{Recuperação de senha}
    \begin{tabular}{|p{1cm}|p{14cm}|}
        \hline
        \multicolumn{1}{|c|}{\textbf{03}} & \textbf{Recuperação de senha}                                                                                                                                                                                   \\ \hline
        \multicolumn{2}{|l|}{\begin{tabular}[c]{@{}l@{}}\textbf{Como} usuário do sistema\\ \textbf{Eu quero} poder solicitar um email de recuperação de senha\\ \textbf{Para que} possa criar uma nova senha para meu usuário.\end{tabular}}                \\ \hline
        \multicolumn{2}{|l|}{\textbf{Critérios de aceitação}}                                                                                                                                                                                               \\ \hline
        \multicolumn{2}{|l|}{\begin{tabular}[c]{@{}l@{}}1. Deve validar se o email informado existe no sistema.\\2. Deve enviar um email de recuperação, com validade máxima de 3 horas.\\3. Deve solicitar a solução de um desafio captcha. \end{tabular}} \\ \hline
        \multicolumn{2}{|c|}{\includegraphics[scale=0.294]{img/resoar-password-recovery.png}}                                                                                                                                                               \\ \hline
    \end{tabular}
\end{table}

\begin{table}[H]
    \caption{Visão geral}
    \begin{tabular}{|p{1cm}|p{14cm}|}
        \hline
        \multicolumn{1}{|c|}{\textbf{04}} & \textbf{Visão geral}                                                                                                                                                                                                                                                                            \\ \hline
        \multicolumn{2}{|l|}{\begin{tabular}[c]{@{}l@{}}\textbf{Como} usuário do sistema\\ \textbf{Eu quero} acessar uma tela de boas vindas ao entrar no sistema, contendo as\\ minhas publicações mais recentes, publicações salvas, e histórico de\\ publicações acessadas.\\ \textbf{Para que} tenha um ponto de partida.\end{tabular}} \\ \hline
        \multicolumn{2}{|l|}{\textbf{Critérios de aceitação}}                                                                                                                                                                                                                                                                               \\ \hline
        \multicolumn{2}{|l|}{\begin{tabular}[c]{@{}l@{}}1. Deve exibir ao menos as 3 últimas publicações do usuário.\\2. Deve exibir ao menos as 3 últimas publicações salvas para leitura.\\3. Deve exibir ao menos um histórico das 3 últimas publicações acessadas. \end{tabular}}                                                       \\ \hline
        \multicolumn{2}{|c|}{\includegraphics[scale=0.294]{img/resoar-overview.png}}                                                                                                                                                                                                                                                        \\ \hline
    \end{tabular}
\end{table}

\begin{table}[H]
    \caption{Minhas publicações}
    \begin{tabular}{|p{1cm}|p{14cm}|}
        \hline
        \multicolumn{1}{|c|}{\textbf{05}} & \textbf{Minhas publicações}                                                                                                                                                                                                                                                                                                                             \\ \hline
        \multicolumn{2}{|l|}{\begin{tabular}[c]{@{}l@{}}\textbf{Como} usuário do sistema\\ \textbf{Eu quero} acessar uma listagem contendo todas as minhas publicações\\ \textbf{Para que} possa navegar pelas publicações que realizei.\end{tabular}}                                                                                                                                              \\ \hline
        \multicolumn{2}{|l|}{\textbf{Critérios de aceitação}}                                                                                                                                                                                                                                                                                                                                       \\ \hline
        \multicolumn{2}{|l|}{\begin{tabular}[c]{@{}l@{}}1. Deve permitir a filtragem de publicações pelo título.\\2. Deve exibir na listagem o título, a imagem de capa, os autores, orientadores e\\ parte do resumo.\\3. Deve possuir um botão para a inclusão de uma nova publicação.\\4. Ao clicar em uma publicação, deve redirecionar para a página detalhada da\\ publicação. \end{tabular}} \\ \hline
        \multicolumn{2}{|c|}{\includegraphics[scale=0.294]{img/resoar-my-research.png}}                                                                                                                                                                                                                                                                                                             \\ \hline
    \end{tabular}
\end{table}

\begin{table}[H]
    \caption{Nova publicação}
    \begin{tabular}{|p{1cm}|p{14cm}|}
        \hline
        \multicolumn{1}{|c|}{\textbf{06}} & \textbf{Nova publicação}                                                                                                                                                                                                                               \\ \hline
        \multicolumn{2}{|l|}{\begin{tabular}[c]{@{}l@{}}\textbf{Como} usuário do sistema\\ \textbf{Eu quero} realizar uma nova publicação\\ \textbf{Para que} possa salvar as minhas publicações no sistema.\end{tabular}}                                                                         \\ \hline
        \multicolumn{2}{|l|}{\textbf{Critérios de aceitação}}                                                                                                                                                                                                                                      \\ \hline
        \multicolumn{2}{|l|}{\begin{tabular}[c]{@{}l@{}}1. Deve solicitar os campos de título, resumo, ano, tipo, visibilidade, idioma,\\ instituição, autores, orientadores e arquivo da publicação.\\2. Dentro do campo de autores sempre deve haver no mínimo o próprio usuário. \end{tabular}} \\ \hline
        \multicolumn{2}{|c|}{\includegraphics[scale=0.46]{img/resoar-add-research.png}}                                                                                                                                                                                                            \\ \hline
    \end{tabular}
\end{table}

\begin{table}[H]
    \caption{Pesquisar publicações}
    \begin{tabular}{|p{1cm}|p{14cm}|}
        \hline
        \multicolumn{1}{|c|}{\textbf{07}} & \textbf{Pesquisar publicações}                                                                                                                                                                                                                                           \\ \hline
        \multicolumn{2}{|l|}{\begin{tabular}[c]{@{}l@{}}\textbf{Como} usuário do sistema\\ \textbf{Eu quero} pesquisar por publicações, podendo utilizar de filtros \\ \textbf{Para que} possa encontrar as publicações que desejo visualizar.\end{tabular}}                                                         \\ \hline
        \multicolumn{2}{|l|}{\textbf{Critérios de aceitação}}                                                                                                                                                                                                                                                        \\ \hline
        \multicolumn{2}{|l|}{\begin{tabular}[c]{@{}l@{}}1. Deve permitir a busca por termos contidos dentro das publicações.\\2. Deve permitir a filtragem por ano, autor, orientador, tipo e idioma.\\3. Uma caixa de pesquisa por publicações deve estar sempre visível\\ na interface do usuário.  \end{tabular}} \\ \hline
        \multicolumn{2}{|c|}{\includegraphics[scale=0.46]{img/resoar-search.png}}                                                                                                                                                                                                                                    \\ \hline
    \end{tabular}
\end{table}

\begin{table}[H]
    \caption{Visualizar publicação}
    \begin{tabular}{|p{1cm}|p{14cm}|}
        \hline
        \multicolumn{1}{|c|}{\textbf{08}} & \textbf{Visualizar publicação}                                                                                                                                                                                                                                                                                                                                                                       \\ \hline
        \multicolumn{2}{|l|}{\begin{tabular}[c]{@{}l@{}}\textbf{Como} usuário do sistema\\ \textbf{Eu quero} visualizar uma publicação \\ \textbf{Para que} possa ver os autores, orientadores, resumo e baixar o arquivo\\ da publicação.\end{tabular}}                                                                                                                                                                                         \\ \hline
        \multicolumn{2}{|l|}{\textbf{Critérios de aceitação}}                                                                                                                                                                                                                                                                                                                                                                                    \\ \hline
        \multicolumn{2}{|l|}{\begin{tabular}[c]{@{}l@{}}1. Deve exibir o título da publicação em destaque.\\2. Deve exibir a capa da publicação, e informações como os autores,\\ orientadores, tipo de publicação, idioma e resumo.\\3. Deve permitir realizar o \emph{download} da publicação.\\4. Deve permitir salvar a publicação para mais tarde.\\5. Deve permitir gerar um \emph{link} de compartilhamento da publicação. \end{tabular}} \\ \hline
        \multicolumn{2}{|c|}{\includegraphics[scale=0.41]{img/resoar-view-research.png}}                                                                                                                                                                                                                                                                                                                                                         \\ \hline
    \end{tabular}
\end{table}

\begin{table}[H]
    \caption{Perfil do usuário}
    \begin{tabular}{|p{1cm}|p{14cm}|}
        \hline
        \multicolumn{1}{|c|}{\textbf{09}} & \textbf{Perfil do usuário}                                                                                                                                                                                                                                                  \\ \hline
        \multicolumn{2}{|l|}{\begin{tabular}[c]{@{}l@{}}\textbf{Como} usuário do sistema\\ \textbf{Eu quero} acessar uma página de perfil de usuário \\ \textbf{Para que} possa atualizar minhas informações, e trocar minha senha.\end{tabular}}                                                                       \\ \hline
        \multicolumn{2}{|l|}{\textbf{Critérios de aceitação}}                                                                                                                                                                                                                                                           \\ \hline
        \multicolumn{2}{|l|}{\begin{tabular}[c]{@{}l@{}}1. Deve exibir o nome e a foto do usuário.\\2. Deve listar as publicações as quais o usuário participa como autor.\\3. Caso o usuário esteja acessando o seu próprio perfil, devem existir opções\\ para editar as informações e trocar a senha. \end{tabular}} \\ \hline
        \multicolumn{2}{|c|}{\includegraphics[scale=0.46]{img/resoar-account.png}}                                                                                                                                                                                                                                      \\ \hline
    \end{tabular}
\end{table}

\chapter*{Conclusão} \label{chap:concl}

Cada vez mais os repositórios institucionais estão sendo adotados por instituições de
ensino superior, com o objetivo de reunir, preservar e oferecer o fácil acesso ao material
científico produzido ao decorrer dos anos. Sendo por meio deste movimento de preservação
digital da produção científica, que iniciativas de acesso aberto se tornam possíveis,
possibilitando que uma produção científica escrita hoje possa ser vista e acessada
de forma gratuita pelo mundo todo, podendo ser utilizada como base ou inspiração para
diversas outras pesquisas.

Também torna-se relevante a existência do desenvolvimento ativo de diferentes
alternativas de \emph{softwares} utilizados para elaborar tais repositórios
institucionais, tendo em vista o constante progresso tecnológico, e a necessidade
de que estas ferramentas não fiquem ultrapassadas ou monopolizadas por uma única
opção de \emph{software}. Além da necessidade da existência de padrões, para que
estas alternativas não criem um ambiente fragmentado de sistemas e padrões que não
são compatíveis entre si.

O objetivo deste trabalho de desenvolver um repositório institucional de pesquisas
acadêmicas baseado em nuvem, visando reunir e preservar as publicações acadêmicas
e científicas produzidas em âmbito universitário, além de unificar o processo
de publicações e correções por parte dos orientadores em uma única plataforma,
foi atingido, por meio do desenvolvimento de um novo \emph{software web} de
repositório institutional que foi denominado de \emph{RESOAR}
(\emph{Research Open Access Repository}).

O \emph{RESOAR} foi desenvolvido tendo como objetivo utilizar de tecnologias baseadas
em nuvem para realizar o armazenamento das publicações acadêmicas, possuindo suporte
a serviços de \emph{Object Storage} como o \emph{Amazon S3} e \emph{Digital Ocean Spaces}.

Além disto, todas as configurações tanto do \emph{backend} quanto do \emph{frontend}
podem ser realizadas por meio de variáveis de ambiente, visando facilitar a implantação
do sistema em ambientes baseados em nuvem por meio de imagens \emph{docker}, tendo como
principio a ideia de nunca precisar acessar remotamente o container para realizar qualquer
tipo de configuração.

A primeira hipótese deste trabalho apresenta que "O recurso de \emph{Full Text Search} presente
no banco de dados PostgreSQL pode ser utilizado como uma alternativa viável para realizar
as consultas por publicações dentro do repositório acadêmico proposto, (menos de 1 segundo
por consulta), mesmo em bases de dados com mais de 3.500 publicações".

Esta primeira hipótese foi validada por meio da execução de testes de carga e estresse
sobre sobre o \emph{endpoint} de consulta por publicações do \emph{backend} desenvolvido.
Com os dados coletados, foi constatado que no cenário descrito pela hipótese, uma consulta
no repositório institucional possui um tempo médio de 114,54 milissegundos.

Já a segunda hipótese deste trabalho afirma que "O processo desenvolvido para extração do
texto das publicações durante o auto arquivamento é rápido o suficiente para não necessitar
de processamento em segundo plano (menos de 3 segundos), mesmo em publicações com mais de
12.000 palavras, cerca de 40 páginas de texto em português com fonte tamanho 12,
considerando que cada página tenha 300 palavras."

Esta segunda hipótese também foi validada por meio da aplicação de um teste de carga
e estresse sobre o \emph{backend} desenvolvido para o repositório institucional. Por meio
das métricas coletadas, foi constatado que uma publicação com as características
descritas pela hipótese, demora em média 2,30 segundos para ser concluída.

Em suma, o foco deste trabalho de desenvolver um novo repositório institucional
de pesquisas acadêmicas, utilizando de tecnologias baseadas em nuvem, e unificar
o processo de publicações e correções por parte dos alunos e orientadores, foi atingido.
Proporcionando aos envolvidos um grande aprendizado nas diversas áreas que envolvem
o desenvolvimento de \emph{software}, além de possibilitar a aplicação em prática dos
conhecimentos adquiridos ao decorrer da graduação.

Também fica como contribuição, a possibilidade da utilização do \emph{software}
desenvolvido, tanto pela SETREM, quanto a outras instituições que venham desenvolver
interesse pela utilização, ou pela contribuição com o projeto por meio de sugestões
ou melhorias ao \emph{software} de repositório institucional desenvolvido.

Como proposta futura, fica a sugestão da criação de um recurso dentro do próprio
repositório institucional, que permita realizar anotações em cima das publicações
acadêmicas, como a adição de comentários ou rabiscos de caneta. Este recurso poderia
ser especialmente útil para os orientadores, que poderiam adicionar sugestões ou
correções as pesquisas dentro da própria plataforma.

No momento em que este trabalho
foi desenvolvido, foram encontradas poucas bibliotecas que permitem este tipo de
edição diretamente pelo o navegador do usuário, sendo em maioria pagas, ou de difícil
implementação, inviabilizando a aplicação do recurso no sistema.

Também como proposta futura, poderia ser realizado uma comparação de desempenho
e utilização de recursos, entre o repositório institucional desenvolvido e as demais
ferramentas existentes.


%\bibliographystyle{template/abnt}
%% para português
\bibliographystyle{template/abnt-pt}
\bibliography{bib/bibliography}

\appendix

\chapter*{Apêndices}

\chapter*{Apêndice A}
De acordo com \citep[p. 128]{LAKATOS2021:metodologia} o orçamento
tem como objetivo responder a questão "quanto será necessário
despender?", distribuindo os gastos em vários items.

\begin{table}[H]
    \caption{Orçamento}
    \label{quad:orcamento}
    \begin{tabular}{|p{6.4cm}|c|r|r|}
        \hline
        {\textbf{Item}}                         & {\textbf{Quantidade}} & {\textbf{Valor Unitário}} & {\textbf{Valor Total}} \\ \hline
        {{Capa}}                                & {2}                   & {R\$ 25,00 }              & {R\$ 50,00}            \\ \hline
        {{Espirais}}                            & {3}                   & {R\$ 5,00 }               & {R\$ 15,00}            \\ \hline
        {{Horas trabalhadas}}                   & {800}                 & {R\$ 30,00 }              & {R\$ 24,000}           \\ \hline
        {{Impressões}}                          & {1000}                & {R\$ 0,30 }               & {R\$ 300,00}           \\ \hline
        {{Serviços de armazenamento em nuvem}}  & {1}                   & {R\$ 26,52 }              & {R\$ 26,52}            \\ \hline
        {{Serviços de Banco de Dados em nuvem}} & {1}                   & {R\$ 79,55 }              & {R\$ 79,55}            \\ \hline
        {{Serviços de hospedagem em nuvem}}     & {1}                   & {R\$ 53,03 }              & {R\$ 53,03}            \\ \hline
        {{Serviços de Designer}}                & {1}                   & {R\$ 350,00 }             & {R\$ 350,00}           \\ \hline
    \end{tabular}
\end{table}

\chapter*{Apêndice B}

Em conformidade com \citep[p. 128]{LAKATOS2021:metodologia} o cronograma
visa responder a pergunta “quando?”, sendo elaborado de forma a
representar a previsão do tempo necessário para realização da pesquisa.

\begin{table}[H]
    \caption{Cronograma}
    \label{quad:cronograma}
    \begin{tabular}{|p{6.85cm}|l|l|l|l|l|l|l|}
        \hline
        \multicolumn{1}{|c|}{}                                      & \multicolumn{7}{c|}{\textbf{2022}}                                                                                                                                                                   \\ \cline{2-8}
        \multicolumn{1}{|c|}{\multirow{-2}{*}{\textbf{Atividades}}} & \textbf{Mai}                       & \textbf{Jun}             & \textbf{Jul}             & \textbf{Ago}             & \textbf{Set}             & \textbf{Out}             & \textbf{Nov}             \\ \hline
        Desenvolvimento do projeto                                  & \cellcolor[HTML]{000000}           & \cellcolor[HTML]{000000} &                          &                          &                          &                          &                          \\ \hline
        Entrega e validação do projeto                              &                                    & \cellcolor[HTML]{000000} &                          &                          &                          &                          &                          \\ \hline
        Desenvolvimento de requisitos funcionais                    &                                    & \cellcolor[HTML]{000000} &                          &                          &                          &                          &                          \\ \hline
        Desenvolvimento de protótipos                               &                                    & \cellcolor[HTML]{000000} & \cellcolor[HTML]{000000} &                          &                          &                          &                          \\ \hline
        Desenvolvimento do repositório acadêmico                    &                                    &                          & \cellcolor[HTML]{000000} & \cellcolor[HTML]{000000} & \cellcolor[HTML]{000000} & \cellcolor[HTML]{000000} &                          \\ \hline
        Organização de Dataset de trabalhos acadêmicos              &                                    &                          &                          &                          & \cellcolor[HTML]{C0C0C0} & \cellcolor[HTML]{000000} & \cellcolor[HTML]{000000} \\ \hline
        Realização de testes no repositório acadêmico               &                                    &                          &                          &                          &                          & \cellcolor[HTML]{C0C0C0} & \cellcolor[HTML]{000000} \\ \hline
        Documentação dos resultados                                 &                                    &                          &                          &                          &                          &                          & \cellcolor[HTML]{000000} \\ \hline
        Entrega do Relatório Final                                  &                                    &                          &                          &                          &                          &                          & \cellcolor[HTML]{000000} \\ \hline
        Apresentação do Relatório Final                             &                                    &                          &                          &                          &                          &                          & \cellcolor[HTML]{C0C0C0} \\ \hline
    \end{tabular}
\end{table}

\begin{table}[H]
    \begin{tabular}{|
            >{\columncolor[HTML]{C0C0C0}}l |l|}
        \hline
                                                        & Proposto  \\ \hline
        \cellcolor[HTML]{000000}{\color[HTML]{000000} } & Realizado \\ \hline
    \end{tabular}
\end{table}

\end{document}
