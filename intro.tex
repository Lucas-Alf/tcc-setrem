\chapter*{Introdução} \label{chap:intro}

Os repositórios acadêmicos dentro de instituições de ensino
podem ser vistos como uma ferramenta para a preservação
e propagação do conhecimento, de forma a reunir o acervo das
pesquisas realizadas durante a passagem dos acadêmicos
pela instituição, disponibilizando fácil acesso a este
material.

Para \cite{LYNCH:2003} um repositório institucional pode
ser definido como um conjunto de serviços disponibilizados
por uma universidade aos seus membros, visando a disseminação
dos materiais digitais produzidos pela instituição, e pelos
membros de sua comunidade.

Todavia, conforme \cite{PORTO:difusao_cientifica_recortes}
muitas instituições de ciência e tecnologia acabam por não
dispor de uma estrutura profissionalizada de comunicação,
suporte e divulgação dos materiais produzidos, não contemplando
a comunicação em seu organograma funcional, e recorrendo a
improvisações na hora da disponibilização de meios de acesso
a divulgação de seus projetos.

Tendo isto em mente, o tema desta pesquisa foi definido como
o desenvolvimento de um repositório institucional de pesquisas
acadêmicas baseado em nuvem, sendo uma continuação direta a um
projeto de prática profissional desenvolvido pelo próprio autor,
durante o sétimo semestre da graduação em Sistemas de Informação, na
Sociedade Educacional Três de Maio - SETREM.

A pesquisa possui como objetivo desenvolver um repositório institucional
de pesquisas acadêmicas projetado para ser executado em ambientes
baseado em nuvem, utilizando de tecnologias como React e Vite para o
frontend, Microsoft .NET 6.0 para a criação de APIs, PostgreSQL como banco
de dados, e recursos de armazenamento de objetos baseados em nuvem.

A plataforma também possui como premissa envolver recursos referentes
a interação entre os alunos e orientadores, englobando o processo de correção
e envio de revisões de publicações dentro da própria plataforma, e que seja
uma alternativa as atuais plataformas de repositórios existentes.

Com a realização desta pesquisa, buscou-se expor a
importância da utilização de repositórios acadêmicos para realizar
a preservação digital dos materiais, evitando assim a perda
ou esquecimento da produção acadêmica realizada dentro da instituição.
Além de trazer contribuição tanto para os acadêmicos da faculdade SETREM, quanto para outras
instituições que possam optar pela utilização do repositório
acadêmico desenvolvido.

Este trabalho foi estruturado da seguinte forma, no capítulo 1
é abordado o plano de estudo, elencando tópicos como a delimitação
do tema da pesquisa, objetivos, justificativa, metodologia, dentre outros.
No capítulo 2 é apresentado o embasamento teórico e conceitual
visto como necessário para realização deste trabalho.

No capítulo 3 são apresentados os resultados obtidos com a realização desta
pesquisa, protótipos e apresentação do repositório acadêmico proposto.
Em última parte, é apresentada a conclusão da pesquisa,
detalhando os objetivos alcançados, validação das hipóteses, dificuldades
e limitações encontradas durante o desenvolvimento, e propostas futuras
de pesquisa.