\chapter*{Introdução} \label{chap:intro}

Os repositórios acadêmicos dentro de instituições de ensino
podem ser vistos como uma ferramenta essencial para a preservação
e propagação do conhecimento, visto que tais repositórios reúnem
o acervo das pesquisas realizadas durante passagem dos acadêmicos
pela instituição, e em teoria disponibilizam fácil acesso a este
material.

Todavia, conforme \cite{PORTO:difusao_cientifica_recortes}
muitas instituições de ciência e tecnologia acabam por não
dispor de uma estrutura profissionalizada de comunicação,
suporte e divulgação dos materiais produzidos, não contemplando
a comunicação em seu organograma funcional, e recorrendo a
improvisações na hora da disponibilização de meios de acesso
a divulgação de seus projetos.

Tendo isto em mente, o tema desta pesquisa surgiu como uma
continuação direta a um projeto de prática profissional
desenvolvido pelo próprio autor, durante o sétimo
semestre da graduação em Sistemas de Informação, na
Sociedade Educacional Três de Maio - SETREM, que possuía
como objetivo analisar o processo de armazenamento e
acesso as publicações realizadas dentro da instituição.

Durante a realização da prática profissional, foi constatado
que a faculdade não possuía um sistema repositório acadêmico
que possibilitasse aos alunos realizar a publicação de seus
artigos e pesquisas. Também foi verificado por meio da aplicação
de um questionário, que mais da metade dos acadêmicos
(cerca de 62,1\% das respostas) apresentavam interesse máximo
em uma escala de 1 a 5, pela utilização de um repositório acadêmico
institucional.

Ao pesquisar sobre repositórios acadêmicos existentes em plataformas
agregadoras como o OpenDOAR\footnote{https://v2.sherpa.ac.uk/opendoar/},
foi constatado que a maioria dos repositórios nacionais utilizam como
base o DSpace, um \emph{software open source} utilizado para a
criação de repositórios acadêmicos institucionais.

Este \emph{software} foi cogitado como um dos principais candidatos
para uma possível implantação dentro da instituição, porem ao verificar
trabalhos relacionados como o de \cite{GarciaRodrigoMoreira2019DdnB},
foi relatado que por se tratar de um \emph{software open source},
fica a critério das instituição todas as questões pertinentes a
atualização de versões, personalização, segurança, privacidade dos
dados e \emph{backups} dos arquivos, oque resulta em um ambiente
fragmentado onde algumas instituições utilizam versões antigas
do DSpace, que por exemplo não funcionam de forma adequada em
dispositivos moveis, conforme relatado no trabalho relacionado
de \cite{FernandesMacedes:2019}.

Possuindo como inspiração os tópicos anteriormente citados,
este trabalho de conclusão de curso possui como objetivo a utilização
de tecnologias baseadas em nuvem para realizar a análise e
desenvolvimento de uma nova plataforma de repositório acadêmico,
que possibilite a interação entre os alunos e orientadores,
englobando o processo de correção e envio de revisões
de publicações dentro da própria plataforma, e que seja uma alternativa
as atuais plataformas de repositórios existentes.

Este artigo foi estruturado da seguinte forma, no capítulo 1
é abordado o plano de estudo, elencando tópicos como a delimitação
do tema da pesquisa, objetivos, justificativa, metodologia, dentre outros.
Já no capitulo 2 é apresentado o embasamento teórico e conceitual
visto como necessário para realização deste trabalho.

No capítulo 3 são apresentados os resultados obtidos com a realização desta
pesquisa, protótipos e apresentação do repositório acadêmico proposto.
Em ultima parte, é apresentado a conclusão da pesquisa,
detalhando os objetivos alcançados, validação da hipóteses, dificuldades
e limitações encontradas durante o desenvolvimento, e propostas futuras
de pesquisa.