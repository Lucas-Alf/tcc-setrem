% Chapter 2

\chapter{Fundamentação Teórica}\label{chap:background}

\section{Gestão do Conhecimento}\label{sec:business}
\subsection{Gestão do Conhecimento Científico}
\subsection{Repositórios Institucionais}
\subsection{Padrão Dublin Core para metadados descritivos}
\subsection{Open Archival Information System (OAIS)}

\section{Ferramentas utilizadas}\label{sec:fundamental}
\subsection{Cloud Computing}
\subsection{Cloud Object Storage}
\subsection{Microsoft .NET}
\subsection{ReactJS}
\subsubsection{Vite}

\subsection{PostgreSQL}
\subsubsection{FullText Search}
\subsubsection{GIN Index}

% \section{Sistemas Relacionados}\label{sec:rs}
% \subsection{DSpace}
% \subsection{Open Journal Systems}
% \subsection{ResearchGate}

\section{Trabalhos Relacionados}\label{sec:rw}

São apresentados neste tópico, os trabalhos relacionados, que possuem
algum ponto em comum com o que se quer realizar no presente projeto.
Pretendesse aqui, identificar estes pontos, e compará-los com a proposta deste projeto.

\subsection{BDTC - Uma Biblioteca Digital de Trabalhos Científicos com Serviços Integrados}

O trabalho desenvolvido por \cite{CERVI:bdtc}, possui como premissa a apresentação
de uma proposta de biblioteca digital de trabalhos científicos, que foi denominada como
BDTC, que possui como objetivo prover o suporte a três pontos fundamentais:
auto-arquivamento do conteúdo, extração de metadados e busca de similaridade.

Como um dos principais diferenciais, foi desenvolvido um mecanismo de
busca por similaridade para as pesquisas realizadas na BDTC,
que permite ao usuário encontrar trabalhos relacionados, mesmo que
parte da palavra pesquisada não seja exatamente igual ao conteúdo presente
no documento.

Para o desenvolvimento deste mecanismo de busca, foi utilizado
um recurso denominado \emph{n-gram}, que permite quebrar uma palavra em
conjuntos de letras de tamanhos variados, é retratado como exemplo o
\emph{3-gram} do termo "Juci", que podem ser quebrado em dois conjuntos de 3
letras, ou seja, \emph{"Juc"} e \emph{"uci"}.

Estes conjuntos são armazenados em uma tabela de índices no banco de
dados, de forma que ao realizar uma consulta a partir de uma palavra,
o sistema retorna todos os trabalhos que contenham algum dos conjuntos
de letras que compõem a palavra pesquisada.

Em relação com o presente projeto, pode ser realizado uma comparação com
a forma como o mecanismo de busca foi desenvolvido na BDTC. Ao envés de
utilizar o recurso de \emph{n-gram}, o sistema proposto neste projeto utilizara
o recurso de \emph{tokenization} presente na ferramenta de \emph{Full Text Search} do
banco de dados PostgreSQL, que possui um resultado final semelhante, porem não
idêntico, visto que o \emph{tokenization} remove o gênero das palavras,
os espaços em branco, palavras comuns e palavras que não são consideradas relevantes.

\subsection{Desenvolvimento da nova Biblioteca Digital da Biblioteca Brasiliana USP: Relato de Experiência}

O relato de experiência desenvolvido por \cite{GarciaRodrigoMoreira2019DdnB}
apresenta o desenvolvimento da na nova plataforma de Biblioteca Digital da
BBM, a Biblioteca Brasiliana Guita e José Mindlin, em forma de retrospectiva
desde o projeto-piloto, relatando os principais problemas, êxitos
e desafios encontrados durante o desenvolvimento do projeto, que envolvia
a digitalização e desenvolvimento de uma coleção digital para a biblioteca.

A Biblioteca Brasiliana Guita e José Mindlin foi inaugurada em março
de 2013, sendo um órgão e entidade acadêmica da Pró-Reitoria de Cultura
e Extensão da USP (Universidade de São Paulo). Este biblioteca envolve o
projeto Brasiliana USP, que foi iniciado em 2005, e tem como objetivo
abrigar a coleção Brasiliana, doada por José Mindlin. Dentro do escopo
deste projeto, em 2008 é iniciado o projeto-piloto da Biblioteca
Brasiliana Digital, que visa a preservação do acervo e democratização
do acesso ao material.

Para o desenvolvimento da biblioteca digital, foi optado por realizar uma
customização do sistema DSpace, um \emph{software open source} de repositório
digital, com recursos como o Djatoka (servidor de imagens) e visualizadores
de livros como IIPImage e BookReader.

Como principais problemas e êxitos, foi ressaltado a forma como as customizações
foram realizadas no DSpace, sendo muitas delas realizadas diretamente no código fonte
do programa, tornando extramente difícil ou até impossibilitando a atualização
para novas versões da plataforma. Resultado em inconsistências na
visualização dos documentos digitalizados, lentidão do sistema, dentro outros
problemas que vieram a surgir ao longo do tempo.

Além disto, também foi relatado a rotativada das equipes como um fator de impacto
para a continuidade do desenvolvimento da biblioteca, que em sua
maioria era constituída por bolsistas, estagiários e poucos profissionais
terceirizados contratados por tempo determinado. Também foi constatado
que as máquinas digitalizadoras adquiridas para a biblioteca não eram
adequadas para o manuseio dos documentos do acervo, visto que os documentos
eram obras raras, e que necessitavam de diversos cuidados para a preservação
e conversação do material.

Com o tempo parte dos problemas foram resolvidos, sendo até adquirido novas
maquinas digitalizadoras, mais modernas e adequadas para o manuseio do material
bibliográfico.

Comparando o trabalho relacionado com o presente projeto de pesquisa,
é possível ressaltar que o atual projeto não tem como intenção a digitalização
de um acervo físico, porem o sistema DSpace que foi utilizado no trabalho
relacionado foi identificado como um sistema muito popular para o desenvolvimento
de repositórios acadêmicos e bibliotecas digitais de universidades, sendo
possível utilizar a experiência adquirida na implantação desse sistema durante
o desenvolvimento do repositório acadêmico proposto.

\subsection{Classificação facetada: proposta de categorias fundamentais para organizar teses e dissertações em uma biblioteca digital}

O artigo desenvolvido por \cite{PereiraClaytonMartins2021Cfpd} tem como
proposta a apresentação de categorias fundamentais, baseadas nos trabalhos
do matemático e bibliotecário indiano S. R. Ranganathan e do \emph{Classification Research Group}, que pode ser utilizadas
durante o desenvolvimento de interfaces de navegação facetada de bibliotecas
digitais e repositórios de dissertações e testes.

Neste trabalho é relatado que é comum em bibliotecas digitais de teses
e dissertações, encontrar problemas referente a facilidade de encontrar
e recuperar documentos neles armazenados, visto que a principal forma de
pesquisa tende a ser por um campo textual, onde é possível efetuar buscas
simples, e em alguns casos utilizar a combinação de operadores lógicos, como
\emph{AND}, \emph{OR} e \emph{NOT}, para buscas mais complexas.

Porem esta forma de pesquisa, por meio de um único campo textual,
exige que o usuário possua conhecimento prévio de noções de logica,
sigla dos cursos, areas e linhas de pesquisa, e que seja capaz de
utilizar essas informações para construir uma busca mais complexa,
geralmente fazendo com que as pesquisas realizadas retornem resultados
vazios, ou não exibam o total potencial de documentos contidos na plataforma,
levando em consideração questões semânticas dos termos utilizados na pequisa.

Desta forma, a abordagem de pequisa facetada permite que o usuário navegue
pela estrutura conceitual das informações armazenadas no repositório, além de
combinar conceitos de diferentes perspectivas ou facetas (janelas ou menus), sendo uma abordagem
de pesquisa mais eficiente, auxiliando o usuário a encontrar o que procura,
de forma visual e intuitiva a partir de palavras chaves modificáveis em um
vocabulário controlado.

Como resultado de sua pesquisa, foram obtidos as seguintes categorias
fundamentais, seguidas de um exemplo: Documento (Trabalho de conclusão), Tipo (Dissertação, Tese);
Curso (Meteorologia); Linha de Pequisa (Sensoriamento);
Tema (Anomalias climáticas); Especialização do tema (Conservação de Energia);
Localização (Oceano Atlântico); e Ano de publicação (2020).

Em relação ao atual projeto de pesquisa, as categorias fundamentais encontradas
no trabalho relacionado poderiam ser utilizadas para a organização dos documentos dentro do repositório
de trabalhos acadêmicos proposto, podendo ser utilizados em recursos de
filtragem das publicações dentro da plataforma.

\subsection{Garantindo acervos para o futuro: Plano de preservação digital para o Repositório Institucional Arca}

A pesquisa realizada por \cite{QueirozArca:2020} tem como objetivo
apresentar o desenvolvimento do plano de ação de preservação digital
para o Arca - Repositório Institucional da Fiocruz, visando descrever
as ações necessárias para garantir a preservação dos documentos, bem como
a adoção de padrões, procedimentos e tecnologias que possam ajudar a garantir
a preservação do seu acervo digital para o futuro.

O repositório institucional da Fundação Oswaldo Cruz (Fiocruz) denominado
Arca, foi criado em 2007 e lançado oficialmente como repositório institucional em 2011,
utilizando como o base o \emph{software open source} Dspace, e tendo como intuito
reunir, hospedar, disponibilizar e dar visibilidade a produção intelectual e cultural
produzida na fundação.

Como padrão referência, o artigo cita o \emph{Open Archival Information System}
(OAIS), presente na norma ISO 14721:2003, e adaptado na normal brasileira
NBR 15472:2007. O OAIS define um modelo para configuração e operação
de um repositório digital de documentos confiável, descrevendo como deve funcionar
a estrutura e fluxo das informações, desde a inserção dos documentos digitais e
metadados, até a forma como ocorre o seu armazenamento e acesso.

Dentre as repensabilidades obrigatórias para atender o modelo OAIS
está a documentação das políticas e procedimentos para garantir
a preservação dos documentos a longo prazo, bem como o plano de ação
de sucessão caso o repositório seja desativado ou substituído por outro.

Como resultado de sua pesquisa, foi elaborado uma estrutura básica
do Plano de Ação de Preservação Digital, onde em primeiro momento
é descrito os elementos essenciais para a preservação, como: o cenário
institucional, a descrição da coleção, avaliação de riscos e ameaças,
e o planejamento das estrategias para prevenção de obsolescência. E em
segundo momento, é optado por uma combinação de estrategias de preservação
a serem aplicadas, como a normalização de formatos de arquivos, e a verificação
periódica do formatos dos arquivos em uso, que possam apresentar riscos
de obsolescência tecnológica.

Como contribuição ao atual projeto de pesquisa, pode ser citado a
apresentação ao padrão OAIS, que pode ser utilizado durante o desenvolvimento
do repositório acadêmico, visando a conformidade com padrões internacionais
de desenvolvimento de repositórios acadêmicos confiáveis, além da apresentação
de um plano de ação de preservação digital, que envolve a sucessão das obras
armazenadas em caso de desativação ou substituição do serviço.

\subsection{O mapeamento dos repositórios institucionais brasileiros: perfil e desafios}

O artigo realizado por \cite{Weitzel:2019} tem como objetivo mapear
os repositórios institucionais brasileiros até o período de maio de 2017,
com a finalidade de identificar a atual situação de conformidade com
a estrategia do Acesso Verde Aberto proposto pela BOIA (\emph{Budapest Open Access Initiative}),
além de contribuir com a orientação de diretrizes nacionais e internacionais
para implementação e desenvolvimento de repositórios institucionais, ou sua integração
em rede.

A BOIA (\emph{Budapest Open Access Initiative}) estabelece duas estratégias
de Acesso Aberto, sendo a primeira o Acesso Aberto Dourado, que se baseia
nos esforços da comunidade cientifica para criar um ambiente ideal onde
os periódicos eletrônicos são disponibilizados sem cobrança de assinaturas
ou taxas impostas pelas editoras, como as APCs (\emph{Article Processing Charges}).
A segunda estratégia é o Acesso Aberto Verde, onde o próprio autor
realiza a publicação de seu periódico, seja a versão inicial ou
final, em um repositório institucional.

Para a realização do estudo, foi realizado um levantamento de dados em
fontes como o OpenDOAR, ROARMAP, ROAR, Lista de Repositórios do IBICT,
edital da FINEP, lista de usuários do DSpace e os repositórios listados
no \emph{The Ranking Web of World Repositories}. Como indicador de
alinhamento com o Acesso Aberto Verde, foi realizada a observação
direta sobre cada um dos repositórios, de forma a verificar se o mesmo
disponibiliza o acesso aos artigos de periódicos.

Como resultado, foi constatado que cerca de 54,5\% dos 101 repositórios
analisados estão alinhados com o Acesso Aberto Verde, concentrando 97,5\%
do total de artigos disponíveis em repositórios brasileiros. Também foi
identificado que os sites agregadores de repositórios mundiais não
expressam a realidade, contendo informações mal catalogadas,
\emph{links} quebrados, dentre outros problemas.

Em relação ao atual projeto de pesquisa, é possível citar a contribuição
por meio da apresentação dos conceitos de Acesso Verde Aberto e Acesso
Dourado, que visam disponibilizar os artigos de periódicos em repositórios
institucionais, sem cobrança de taxas ou assinaturas. Também é possível
ressaltar a apresentação de agregadores como OpenDOAR, onde é possível
verificar que majoritariamente os repositórios brasileiros são
desenvolvidos em cima do \emph{software} DSpace.

\subsection{Encontrabilidade da informação no repositório institucional da Unesp: um estudo de eye tracking em dispositivos móveis}

A dissertação de mestrado elaborada por \cite{FernandesMacedes:2019}
possui como objetivo compreender a forma como ocorre a encontrabilidade
da informação em repositórios institucionais a partir do uso de dispositivos
moveis, tendo como base o Repositório Institucional da UNESP.

Para o desenvolvido da pesquisa foi utilizado do método quadripolar, onde
no polo epistemológico foi realizado a definição do objetivo da pesquisa,
no espoco da Ciência da Informação; no polo teórico é realizado a
fundamentação conceitual sobre repositórios digitais, encontrabilidade
da informação e dispositivos moveis; no polo técnico foi utilizado um
conjunto de metodologias com \emph{checklist}, teste com \emph{eye tracking},
e entrevistas; e no polo morfológico é realizado a apresentação dos resultados.

Por meio da pesquisa realizada, foi constatado que grande parte dos
repositórios são criados com base no \emph{software open source} DSpace,
que em suas versões mais recentes já possui uma interface responsiva,
sendo adequada a diferentes tamanhos de telas. Porem, visto que as
atualizações do \emph{software} DSpace dependem da equipe técnica das instituições,
muitos repositórios acadêmicos se encontram desatualizados, tornando
difícil a navegação via dispositivos moveis.

Neste ponto é possível traçar uma correlação ao trabalho desenvolvido por
\cite{GarciaRodrigoMoreira2019DdnB}, onde é constatado que as
personalizações realizadas pelas instituições no DSpace,
podem acabar dificultando ou até impedindo a atualização do
\emph{software} para novas versões.

Como resultados de sua pesquisa, foi constatado que o Repositório
Institucional UNESP já possui algum nível de preocupação com a
responsividade para dispositivos moveis em seu \emph{website},
entretanto com a pesquisa realizada foram encontrados alguns
pontos que podem ser aprimorados.

Entre os tópicos que podem ser aprimorados no repositório,
um dos principais seria a falta da caixa de busca no corpo
da página, que se encontra oculto dentro de um menu. A sugestão
seria que esta caixa de busca esteja visível logo no primeiro
momento, pois é um dos elementos principais dentro do repositório.

Sendo assim, foi possível afirmar que no repositório analisado,
elementos não relevantes em dispositivos moveis possuem maior destaque
do que outros elementos que possuem maior importância,
como por exemplo a função de auto-arquivamento,
sendo um recursos muito pouco utilizado a partir de dispositivos moveis,
e que possui uma visibilidade maior que a caixa de pesquisa.

Outros recursos que podem ser aprimorados no repositório seriam
o recurso de autocompletar na caixa de busca, facilitando a localização
dos conteúdos pesquisados. E o recurso de \emph{wayfinding} poderia ser
melhorado, mudando a cor de \emph{links} já visitados, componentes
de \emph{breadcrumb} alternativos para dispositivos moveis, e
\emph{link} visível para retornar a pagina inicial.

Como contribuição ao trabalho atual, é possível ressaltar o resultados
obtidos com a análise de \emph{eye tracking} realizado pela pesquisa,
podendo ser utilizada para a criação de uma interface gráfica para
o repositório acadêmico, que exiba conteúdos mais relevantes para
os usuários de dispositivos moveis.

\subsection{The Emergence of Institutional Repositories: A Conceptual Understanding of Key Issues through Review of Literature}

Na pesquisa realizada por \cite{Saini:2018} é realizado uma revisão de
literatura sobre o surgimento dos repositórios institucionais,
buscando a compreensão conceitual dos principais problemas que envolvem
a criação e uso de repositórios institucionais. A revisão de literatura
consiste em uma organização conceitual da combinação de resultados que provem
contexto para pesquisas. Ajudando a refinar ideias, conhecer especificações do
método de pesquisa, e trazer claridade sobre diversos fatores que cercam o
surgimento, motivação, custos, seleção de \emph{software} e uma perspectiva global
de sobre os repositórios institucionais.

Em sua pesquisa, foi constatado que o surgimento dos repositórios institucionais
não consiste em um fenômeno recente em instituições acadêmicas, e que este tema já vem
existindo por quase uma década, surgindo primeiramente em bibliotecas de instituições
de grande porte, que realizaram o papel de \emph{early adopters} no desenvolvimento
de repositórios institucionais mais focados na acumulação, preservação e disseminação
da pesquisa realizada na instituição, de uma forma acessível e aberta.

Como motivação para a adoção de repositórios institucionais, é considerado
que estes fazem parte da plataforma de serviços oferecidos pela instituição,
sendo um recurso que pode ser utilizado para vender e divulgar o nome da instituição,
de forma a propagar as suas pesquisas e facilitar o intercambio de conhecimento.

Já sobre os fatores de custos, é possível elencar um elevado número de considerações,
como a utilização de \emph{software open source} ou o licenciamento de \emph{software}
proprietário, a abertura de acesso fora do campus, ou o fornecimento de acesso total, ou
limitado do conteúdo. Também foi relatado que dentre os maiores está o custo operacional
da biblioteca, e que ao optar pelo auto arquivamento digital, boa parte dos custos
podem ser evitados.

Sobre a atuação de \emph{softwares open source}, foi relatado que os dois sistemas mais
predominantes são o Eprints e o DSpace. O Eprints foi lançado em 2000, desenvolvido por
Stephen Harnad e sua equipe na Universidade de Southampton, já o DSpace foi lançado 2002
pelo Instituto de Tecnologia do Massachusetts (MIT). O Eprints foi projetado para suportar
formais mais tradicionais de publicações, como periódicos, conferencias, artigos, capítulos
de livros, e outros. Já o DSpace tem como intenção suportar uma variedade muito maior de
documentos, como comunicados formais da instituição, e outras categorias de literatura.

Por fim, sobre a perspectiva global de repositórios institucionais, foi relatado que
a maior parte se encontra na Europa, contribuindo com 47.92\% do total,
seguido pela América do Norte com 29.28\%, Asia com 11.04\%, Austrália com 5.84\%,
a América do Sul com 4.40\%, e Africa com 1.52\% dos repositórios mundiais, tendo como fonte
dados de março de 2018 presentes no OpenDoar.

Este artigo pode ser visto como relacionado, visto que atua como uma revisão de literatura
sobre o surgimento dos repositórios institucionais, que pode ser utilizada como embasamento
para a atual pesquisa, trazendo dados referentes a motivação, custos e \emph{softwares} existes.

\subsection{Why So Many Repositories? Examining the Limitations and Possibilities of The Institutional Repositories (IR) Landscape}

No trabalho realizado por \cite{Arlitsch:2018} possuir como objetivo examinar as limitações
e possibilidades encontradas no cenário de repositórios institucionais. Em sua pesquisa
foi identificado que o número de repositórios vinha crescendo lentamente até meados de 2005,
período em ocorreu um grande aumento no número de repositórios existentes, de 128 em dezembro
de 2005 para 2,253 em dezembro de 2012 conforme dados do OpenDoar, representando um aumento
de 1660\% durante o período.

Todo via, grande parte dos repositórios que surgiram ao longo do tempo são hospedados e
mantidos de forma local pela própria instituição, oque até possui alguns aspectos positivos,
como a flexibilidade, controle, customização, e a tendência de vários sistemas distribuídos
possuírem menor risco de falha massiva que um único sistema centralizado.

Porem também existem aspectos negativos sobre a existência de múltiplas instâncias
de repositórios acadêmicos sendo hospedados em vários locais, por diferentes
instituições, como por exemplo a pesquisa e encontrabilidade do conteúdo, problemas
com a coleta e armazenamento dos metadados, e a variedade de diferentes versões de
\emph{software} sendo utilizadas.

É relatado que a fragmentação da utilização de diferentes versões de \emph{software}
pela comunidade é um tópico dramático, citando como exemplo o \emph{DuraSpace Registry}
que realiza a listam dos repositórios que utilizam como base o DSpace, foi identificado
cerca de 2,004 repositórios que estavam rodando a plataforma DSpace até dezembro
de 2017, destes cerca de 800 (quase 40\%) estavam utilizando a versão 1.8 ou
menor, uma versão lançada em 2011, sendo que qualquer versão abaixo de 5.x não é
mais suportada desde janeiro de 2018.

As implicações negativas desta situação não podem ser ignoradas pelas instituições,
visto que a cada nova versão de \emph{software} que é lançada, fica mais desafiador
para as instituições realizar a atualização destes sistemas. Além da sucessão a
riscos de segurança e ataques de ransomware, que comprometem a confiança que a comunidade
possui em depositar suas publicações nestes repositórios.

Em sua pesquisa é citado que a ideia das instituições hospedarem seu próprio repositório
institucional de forma local já está obsoleta, e deveriam ser substituídos por alternativas viáveis,
fazendo uma relação ao tempo que as instituições também hospedavam os seus próprios servidores de
email, onde hoje em dia são utilizados serviços baseados em nuvem como os fornecidos pela
Google e Microsoft.

Porem o conceito de alternativa viável pode cobrir muitas possibilidades,
e não existe um acordo comum sobre uma escolha correta, algumas discussões focam na adoção de versões
mais recentes de sistemas como o DSpace e Fedora Commons, outros focam em soluções comerciais
baseadas em nuvem, que oferecem escalabilidade e redundância.

Como conclusão, é estimado que a proliferação de repositórios locais por bibliotecas de
instituições acadêmicas, criam mais problemas do que trazem benefícios, considerando
que existem algumas vantagens de possuir um controle local, porem a existência de
centenas de repositórios diferentes não servem mais ao usuário comum, e acabam trazendo
diversos problemas e custos.

Este trabalho pode ser visto como relacionado a presente pesquisa, visto que aborda
tópicos relacionados as vantagens e desvantagens de possuir um repositório local, ou
investir em soluções baseadas em nuvem que podem ser fornecidas como serviços a instituição.

\subsection{Next generation Institutional Repositories: The case of the CUT Institutional Repository KTISIS}

O trabalho relacionado de \cite{Zervas:2019} possui como foco a transformação
do repositório institucional da \emph{Cyprus University of Technology} (CUT) denominado
KTISIS, em um \emph{Current Research Information System} (CRIS), sendo a primeira
apresentação de uma implementação europeia baseada no \emph{software open source}
DSpace-CRIS.

A motivação para a elaboração da pesquisa consiste em atender as necessidades
dos pesquisadores, permitindo que possam submeter seus trabalhos e perfil acadêmico
para o KTISIS. Tendo como alteração do KTISIS mais notável a provisão do Perfil do Acadêmico,
onde os pesquisadores terão acesso dedicado a um conjunto de funcionalidades que
agregam valor aos seus trabalhos, e ao próprio repositório.

Os trabalhos realizados no KTISIS possuem como objetivo seguir as diretrizes e
os comportamentos sugeridos pela \emph{Next Generation Repositories} publicado
pelo COAR (\emph{Confederation of Open Access Repositories}), que visam
colocar os repositórios em posição para a fundação de uma infraestrutura de rede
globalmente distribuída para a comunicação acadêmica.

Como principais benefícios da transformação do repositório KTISIS em um sistema
CRIS usando o DSpace-CRIS, está a provisão de um modelo de dados flexível
que permite descrever todas os documentos que compõem o ambiente acadêmico e suas
ligações, por meio da provisão de identificadores únicos que permitem criar relações
entre os documentos do sistema, como por exemplo o ORCID (\emph{Open Researcher and Contributor ID})
que seria o identificador do pesquisador. Também foi realizado o uso da
funcionalidade que permite importar registros de bases de dados externas
como a Web of Science e o Scopus.

Como contribuição a atual pesquisa, é possível constatar a apresentação dos
conceitos de \emph{Next Generation Repositories} que visa fornecer diretrizes
e sugestões para o desenvolvimento de repositórios acadêmicos preparados
para uma rede global de comunicação acadêmica. Alem dos conceitos de identificadores
únicos para relacionar e citar documentos e autores no repositório como o ORCID.

\subsection{Using Open Access Institutional Repositories to Save the Student Symposium during the COVID-19 Pandemic}

O trabalho relacionado de \cite{Symulevich:2022} tem como objetivo
descrever como duas universidades utilizaram de seus repositórios
institucionais para adaptar os seus simpósios de pesquisa acadêmica
a eventos virtuais em questão de semanas.

O primeiro caso traz o relatado de como a mostra de primavera para pesquisa e
inquérito criativo da universidade de Longwood teve de ser adaptada para um
evento online. A universidade de Longwood consiste em uma universidade publica
do centro-sul da Virginia nos Estados Unidos, que desde a primavera de 2018
realiza um simpósio de pesquisa acadêmica de alunos dos cursos de graduação,
de forma presencial.

Com o crescimento dos casos de COVID-19, os administradores da instituição
começaram a discutir a possibilidade de alterações na forma como os eventos
são realizados, sendo constatado que o repositório acadêmico da universidade
poderia ser utilizado para hospedar as publicações do evento.

Após a aprovação da ideia, o diretório de pesquisas contatou a biblioteca da universidade sobre
a possibilidade da utilização do repositório Bepress Digital Commons\footnote{https://bepress.com/},
para hospedar as publicações do evento virtual, além de requisitar ferramentas
para facilitar a comunicação entre os acadêmicos por meio e de comentários
que são realizados durante a apresentação do evento de tres dias.

Dentre as ferramentas, foram elencados o Disquis\footnote{https://disqus.com/}
e o Intense Debate\footnote{https://intensedebate.com/} como ferramentas para troca
de comentários. Também foram discutidas soluções para o envio de apresentações
no formato PowerPoint que continham videos embutidos, visto que o arquivo do video
é separado do arquivo da apresentação.

A Universidade de Longwood realizou outros dois eventos no formatado virtual,
no outono de 2020 e na primavera de 2021. Os organizadores optaram por migrar
para a plataforma Symposium by Forager One\footnote{https://symposium.foragerone.com/}
para a realização dos eventos.

O segundo caso aborda a realização do simpósio virtual de pesquisa acadêmica
da \emph{University of South Florida St. Petersburg}, sendo uma ramificação
da \emph{University of South Florida} (USF). Com a chegada da COVID-19,
a universidade teve de se adaptar ao ambiente virtual, e pausar as atividades
presenciais no campus. Com isto o escritório de pesquisa começou a entrar em contato
com os bibliotecários da \emph{Nelson Poynter Memorial Library} para discutir
a possibilidade de um simpósio virtual.

Uma variedade de plataformas foram elencadas como candidatas para realizar
a hospedagem do simpósio, como o Website do campus, Canvas, LibGuides,
Bepress Digital Commons, e até o Facebook. Porem levando como critério fatores
relacionados a moderação, segurança, engajamento e arquivamento, foi optado
pela recomendação do próprio repositório da instituição para hospedar o
simpósio virtual.

Como o escritório de pesquisa queria manter uma experiência semelhante
ao evento presencial, foi requisitado que a plataforma incluísse tanto opções
de video quanto de audio, publicação e visualização de posters, e participação
por meio de comentários.

A biblioteca teve de realizar upload de 55 projetos de pesquisa, 43 deles
contendo uma apresentação em audio ou video. Para realizar o upload dos arquivos
foi optado por elaborar uma planilha de metadados como o nome dos autores,
títulos das publicações, abstract's e links para os arquivos de audio e video,
permitindo o upload em massa das publicações de uma única vez.

Como o repositório da instituição era feito em com base no Bepress Digital
Commons, sistema que não possui suporte a troca de comentários, o time de
tecnologias optou por integrar a ferramenta Intense Debate ao repositório
para poder cumprir com o requisito.

Este pesquisa pode ser considera relacionada ao atual trabalho, uma vez
que relata os desafios enfrentados pelas instituições na tentativa de
hospedar seus simpósios virtuais dentro de repositórios institucionais.
Além de citar tecnologias e recursos que foram vistos como necessários
para a realização dos eventos.

\subsection{Understanding Institutional Repository in Higher Learning Institutions: A Systematic Literature Review and Directions for Future Research}

A revisão de literatura desenvolvida por \cite{Asadi:2019} possui como objetivo
proporcionar um melhor entendimento e uma revisão aprofundada sobre o atual estado
dos estudos envolvendo IRs. A pesquisa utiliza o método de revisão sistemática
de literatura e segue um protocolo para organizar de forma apropriada os
trabalhos relatados a repositórios institucionais.

Os dados foram coletados de estudos publicados entre 2007 e 2018, em
seis das maiores bases de dados (ScienceDirect, IEEE Explorer, Springer, ACM,
Taylor and Francis e Emerald Insight), seguindo um critério de revisão, aplicação
a inclusão da pesquisa ou exclusão, com um total de 115 estudos sendo
incluídos na principal parte da pesquisa.

Como resultado da analise dos estudos, foi constatado a falta de conhecimento
sobre repositórios de acesso aberto por escolas e instituições de ensino, e
a infraestrutura inadequada tanto de informação quanto de comunicação, formam
desafios significantes para o desenvolvimento de repositório de acesso aberto.
Também identificado que dentre os maiores benefícios da adotação de repositórios
institucionais estão a visibilidade aumentada da instituição, o aumento em
rankings locais e globais, aumento do prestigio e valor publico, a melhora do
ensino, aprendizado e pesquisa desenvolvida nas instituições.

Também foi relatado que a maior parte dos estudos sobre esta área de pesquisa
focam na ``implantação, implementação e adoção`` e os ``benefícios e desafios``
que tais repositórios trazem as instituições.

Este trabalho de revisão de literatura pode ser considerado relacionado
a atual pesquisa, pois visa trazer embasamento sobre o atual estado
dos repositórios acadêmicos, além de diretrizes para a implementação
de IRs em instituições de ensino.

\subsection{Discovery Tools to Enhance Resources Findability in the Institutional Repositories: an overview}

A pesquisa realizada por \cite{Mettai:2019} possui como objetivo demonstrar
formas como os repositórios institucionais podem melhorar a recuperação e
descoberta por recursos, de forma a melhor suprir as necessidades dos usuários,
por meio da realização de uma analise das interfaces de um conjunto
de repositórios digitais das universidades da Argélia, possuindo como base
a literatura documentada para trazer fundamentação teórica a analise realizada.

Para a realização da pesquisa, foi realizada uma revisão da literatura
envolvendo temas como o surgimento de ferramentas de descoberta, o comportamento
das pesquisas, e validação das ferramentas de pequisa como parte essencial da
biblioteca e repositórios digitais.

No paper é definido uma ferramenta de descoberta (\emph{Discovery Tool}) como
sendo um programa de computador que trabalha como um mecanismo de busca
compreensivo para melhorar a encontrabilidade de recursos, de forma a coletar
todos os metadados de diferentes locais e reuni-los em um único local que
permite pesquisas abrangentes. Também é definido que em geral, estas ferramentas
consistem de dois componentes principais, o primeiro realiza a unificação
e indexação, e o segundo fornece uma camada de descoberta de conteúdo que
provem diferentes recursos como o ranking de relevância, interface intuitiva
e pesquisa facetada.

Em sua revisão de literatura, foi constatado que a maior parte dos usuários
não se sentem confortáveis com a utilização de ferramentas de descoberta de
pesquisa fornecidos por bibliotecas, pois muitos usuários já possuem uma ideia
formada sobre tais ferramentas de pesquisa. Logo quando o Google Scholar
foi lançado, muitos usuários o aderiram de forma imediata, visto que este oferece
uma experiência de pesquisa muito melhor que as interfaces providas por bibliotecas.
Por outro lado, também foi verificado que em geral os usuários se sentem
satisfeitos, e até mesmo preferem utilizar uma base de dados especializada
para temas de pesquisa específicos, ou o Google Scholar para pesquisa de
próstio mais geral.

Em relação ao futuro das ferramentas de descoberta,
é citado 12 recursos que são vistos como essenciais, sendo um local
unificado de pesquisa para toda a informação, uma interface web
que esteja no estado da arte, o enriquecimento de conteúdo, navegação
facetada, campo de pesquisa por palavras chaves em todas as páginas,
ordenação de relevância, mecanismo de correção de texto como ``Você quis dizer...``
e a recomendação de materiais relacionados.

Como contribuição a atual pesquisa, é possível citar os recursos mencionados
como essenciais para ferramentas de descoberta, que também podem ser aplicados
no contexto de repositórios institucionais. Além da apresentação da relevância
de que as pesquisas contidas no repositório devem ser indexadas por mecanismos
como o Google Scholar para ganhar maior visibilidade.

\subsection{Crafting Linked Open Data to Enhance the Discoverability of InstitutionalRepositories on the Web}

O trabalho relacionado de \cite{Jin:2019} realiza um estudo de como o BIBFRAME 2.0
(\emph{Bibliographic Framework}) pode ser utilizado para descrever objetos em repositórios institucionais, tendo como
objetivo trazer esforços conjuntos entre duas comunidades devotadas ao
conhecimento aberto. No estudo é examinado um conjunto de mapeamentos e conversões
do padrão Dublin Core para o BIBFRAME 2.0, para verificar se o novo padrão irá
trazer ganhos na visibilidade dos documentos digitais na web.

No artigo é definido o conceito de \emph{Linked Open Data} como sendo um
método de publicação de dados estruturados, para que estes possam ser
interconectados, de forma a utilizar a rede para conectar pedaços de dados,
informação e  conhecimento dentro da rede semântica (\emph{Semantic Web}), utilizando de URIs e
RDF, normalmente usando de uma licença aberta, que não impeça a utilização
de forma gratuita.

Já o BIBFRAME é delimitado como sendo um modelo de dados para a descrição
bibliográfica, que foi desenvolvido para substituir o padrão MARC, que
era utilizado para acomodar comunidades de usuários amplas como as de museus,
arquivos, e editoras por meio da utilização de princípios de dados conectados,
para tornar os dados bibliográficos mais uteis tanto para a comunidade interna
quanto externa. O BIBFRAME e os conceitos de dados conectados permitem que
os recursos bibliográficos sejam publicados de forma que a \emph{Web} possa
entender, tornando mais fácil encontrar o conteúdo em pesquisas
e buscas realizadas na \emph{Web}.

Para sumarizar o processo de migração, foram extraídas 459 teses do
repositório institucional, após foi realizado a conversão do padrão
Dublin Core para o MARCXML, usando a ferramenta MarcEdit. Em seguida
foi convertido o padrão MARCXML para o BIBFRAME, por meio da ferramenta
\emph{Library of Congress MARCXML to BIBFRAME Transformation software}.

Como conclusão, foram relatados problemas ao realizar a transição do
padrão Dublin Core para o BIBFRAME 2.0, como por exemplo a baixa
qualidade dos metadados no repositório, e outras dificuldades ao
mapear registros entre um modelo e outro, citando que o BIBFRAME 2.0
possui algumas dificuldades em mapear registros não baseados no padrão
MARC.

E possível dizer que este trabalho se relaciona a atual pesquisa, ao
abordar diferentes padrões de armazenamento de metadados que são essenciais
para o desenvolvimento do repositório acadêmico proposto, além de trazer
algumas comparações entre o padrão Dublin Core e o BIBFRAME 2.0.



\pagebreak

\begin{landscape}
    \begin{table}[H]
        \captionsetup{justification=centering}
        \caption{Trabalhos Relacionados}
        \centering
        \resizebox{1.58\textwidth}{!}{% 
            \begin{tabular}{|p{5cm}|p{10cm}|p{8cm}|p{10cm}|p{12cm}|}
                \hline
                \textbf{Trabalho}                    & \textbf{Objetivo}                                                                                                                                                                                                     & \textbf{Metodologia}                                                                                        & \textbf{Resultados}                                                                                                                                                                                   & \textbf{Conclusão}                                                                                                                                                                                                                         \\ \hline
                \cite{PereiraClaytonMartins2021Cfpd} & Apresenta uma proposta de categorias fundamentais a serem utilizadas para a construção de uma interface de navegação facetada para uma biblioteca digital                                                             & Qualitativa e exploratória.                                                                                 & Foram criadas 8 categorias fundamentais para a navegação facetada.                                                                                                                                    & As categorias fundamentais propostas neste trabalho podem contribuir significativamente para a construção de um mecanismo de busca facetada.                                                                                               \\ \hline
                \cite{QueirozArca:2020}              & Construção do Plano de Ação de Preservação Digital para o Arca – Repositório Institucional da Fiocruz                                                                                                                 & Qualitativa e descritiva                                                                                    & O Plano de Ação de Preservação Digital do repositório Arca estabelece os padrões que visam garantir que a produção cientifica seja preservada de forma permanente, em um ambiente confiável e seguro. & Algumas das recomendações resultantes dos diagnósticos não estão diretamente relacionadas ao Plano de Ação de Preservação Digital, mas podem ser incorporadas ao Plano Operativo do repositório Arca.                                      \\ \hline
                \cite{GarciaRodrigoMoreira2019DdnB}  & Desenvolvimento da na nova plataforma de Biblioteca Digital da BBM (Biblioteca Brasiliana Guita e José Mindlin)                                                                                                       & Descritiva e relato de experiência.                                                                         & Foi optado por realizar uma customização do sistema DSpace, um \emph{software open source} de repositório digital.                                                                                    & Como principais problemas e êxitos, foi ressaltado  que a forma como as customizações foram realizadas no DSpace tornam difícil para novas versões da plataforma.                                                                          \\ \hline
                \cite{Weitzel:2019}                  & Mapear os repositórios institucionais brasileiros até o período de maio de 2017 a fim de retratar a situação atual e contribuir com subsídios para orientar as ações e diretrizes para implementação de repositórios. & Levantamento exaustivo de repositórios brasileiros por meio de fontes específicas e pela observação direta. & Foi verificado que cerca de 54,5\% dos repositórios concentram 97,5\% do total de artigos dentre os 101 repositórios identificados no país.                                                           & O estudo recomenda que o governo brasileiro estabeleça diretrizes nacionais para fomentar as boas práticas em repositórios no país com o objetivo de fortalecer o desenvolvimento do Acesso Aberto Verde                                   \\ \hline
                \cite{FernandesMacedes:2019}         & Compreender de que forma ocorre a encontrabilidade da informação em repositórios institucionais a partir do uso de dispositivos móveis, com ênfase no Repositório Institucional UNESP                                 & Método quadripolar.                                                                                         & No repositório analisado elementos não relevantes em dispositivos moveis possuem maior destaque do que outros elementos que possuem maior importância.                                                & O Repositório Institucional UNESP já possui algum nível de preocupação com a responsividade para dispositivos moveis em seu \emph{website}, entretanto com a pesquisa realizada foram encontrados alguns pontos que podem ser aprimorados. \\ \hline
                \cite{CERVI:bdtc}                    & Apresentação da proposta de biblioteca digital de trabalhos científicos, que foi denominada como BDTC.                                                                                                                & Qualitativa                                                                                                 & Apresentação uma biblioteca digital de trabalhos científicos, a BDTC, que faz uso de serviços como autoarquivamento de conteúdo, extração de metadados dos objetos digitais e busca por similaridade. & Com relação ao processo de extração automática dos metadados dos objetos digitais, a ocorrência de erros é minimizada, ficando a critério do autor realizar apenas a confirmação dos metadados extraídos.                                  \\ \hline
            \end{tabular}%
        }
    \end{table}
\end{landscape}
