% Chapter 2

\chapter{Fundamentação Teórica}\label{chap:background}


\section{Área dos Negócios}\label{sec:business}


\section{Fundamentos da Área}\label{sec:fundamental}

Equação ~\ref{eq:my_equation} é um exemplo de uma equação em Latex:


\begin{equation}\label{eq:my_equation}
    h_t = f(W^{(hh)}h_{t-1} + W^{(hx)}x_t).
\end{equation}


Figura\ref{fig:diagram} é um exemplo para incluir uma figura.

\begin{figure}[htb]
    \caption{Simple diagram}
    \centering
    \includegraphics[scale=1.9]{img/diagram.pdf}
    \label{fig:diagram}
    \source{Retirado de \cite{larcc}.}
\end{figure}

\cite{GRIEBLER:IJPP:18}


\cite{MACCOOL:structured_patterns:book:12}


\section{Trabalhos Relacionados}\label{sec:rw}

São apresentados neste tópico, os trabalhos relacionados, que possuem
algum ponto em comum com o que se quer realizar no presente projeto.
Pretendesse aqui, identificar estes pontos, e compará-los com a proposta deste projeto.

\subsection{BDTC - Uma Biblioteca Digital de Trabalhos Científicos com Serviços Integrados}

O trabalho desenvolvido por \cite{CERVI:bdtc}, possui como premissa a apresentação
de uma proposta de biblioteca digital de trabalhos científicos, que foi denominada como
BDTC, que possui como objetivo prover o suporte a três pontos fundamentais:
auto-arquivamento do conteúdo, extração de metadados e busca de similaridade.

Como um dos principais diferenciais, foi desenvolvido um mecanismo de
busca por similaridade para as pesquisas realizadas na BDTC,
que permite ao usuário encontrar trabalhos relacionados, mesmo que
parte da palavra pesquisada não seja exatamente igual ao conteúdo presente
no documento.

Para o desenvolvimento deste mecanismo de busca, foi utilizado
um recurso denominado \emph{n-gram}, que permite quebrar uma palavra em
conjuntos de letras de tamanhos variados, é retratado como exemplo o
\emph{3-gram} do termo "Juci", que pode ser quebrado em dois conjuntos de 3
letras, ou seja, \emph{"Juc"} e \emph{"uci"}.

Estes conjuntos são armazenados em uma tabela de indices no banco de
dados, de forma que ao realizar uma consulta a partir de uma palavra,
o sistema retorna todos os trabalhos que contenham algum dos conjuntos
de letras que compõem a palavra pesquisada.

Em relação com o presente projeto, pode ser realizado uma comparação com
a forma como o mecanismo de busca foi desenvolvido na BDTC. Ao envés de
utilizar o recurso de \emph{n-gram}, o sistema proposto neste projeto utilizara
o recurso de \emph{tokenization} presente na ferramenta de Full Text Search do
banco de dados PostgreSQL, que possui um resultado final semelhante, porem não
idêntico, visto que o \emph{tokenization} remove o gênero das palavras,
os espaços em branco, palavras comuns e palavras que não são consideradas relevantes.

\subsection{Desenvolvimento da nova Biblioteca Digital da Biblioteca Brasiliana USP: Relato de Experiência}

\subsection{Classificação facetada: proposta de categorias fundamentais para organizar teses e dissertações em uma biblioteca digital}

