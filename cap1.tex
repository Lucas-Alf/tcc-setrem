\chapter{Plano de Estudo e Pesquisa} \label{chap:ResearchPlan}

\section{Tema} \label{sec::Theme}
Desenvolvimento de um Repositório Institucional de Periódicos Acadêmicos
baseado em nuvem.

\subsection{Delimitação do Tema} \label{subsec::ThemeDelimitation}

A delimitação do tema se dará como o desenvolvimento de um
repositório institucional de trabalhos acadêmicos, utilizando tecnologias
baseadas em nuvem para realizar o armazenamento e acesso aos dados,
e que envolva o processo de correções e envio de revisões de
publicações dentro da própria plataforma.

O desenvolvimento deste projeto de pesquisa foi realizado durante o período de
maio a julho de 2022, pelo acadêmico Lucas Machado Alf do curso bacharelado
em Sistema de Informação na Sociedade Educacional Três de Maio – SETREM.

\section{Objetivo Geral} \label{sec:objective}

Desenvolver um repositório institucional de periódicos acadêmicos
baseado em nuvem, tendo como objetivo reunir e preservar
as publicações acadêmicas e científicas produzidas em âmbito universitário,
além de unificar o processo de publicações e correções por parte dos orientadores
em uma única plataforma.

\subsection{Objetivos Específicos}
\begin{enumerate}
      \item Explorar ferramentas existentes de repositório acadêmico.
      \item Pesquisar sobre trabalhos e artigos relacionados.
      \item Pesquisar sobre formas de armazenamento de documentos em nuvem.
      \item Investigar mecanismos de busca e análise textual de documentos. (Ex. ElasticSearch e Full Text Search).
      \item Elicitar os requisitos para o desenvolvimento do repositório acadêmico.
      \item Definir as tecnologias que serão utilizadas para o desenvolvimento (Ex: banco de dados, linguagens e frameworks).
      \item Desenvolver uma aplicação web de repositório acadêmico.
      \item Realizar testes no repositório acadêmico desenvolvido, verificando o desempenho da aplicação em relação ao crescimento da quantidade de registros publicados na plataforma.

\end{enumerate}


\section{Justificativa}\label{sec:justification}

O tema desta pesquisa surgiu como uma continuação direta a um projeto
de Prática Profissional desenvolvido pelo próprio autor, durante o sétimo
semestre da graduação em Sistemas de Informação, na Sociedade Educacional
Três de Maio – SETREM, que tinha como objetivo analisar o atual processo
de armazenamento e acesso as publicações produzidas dentro da faculdade,
e realizar a representação gráfica dos processos por meio de diagramas BPMN
(\emph{Business Process Model and Notation}).

Conforme o próprio autor, durante o desenvolvimento da pesquisa foi constatado que a faculdade
não possuía um sistema de repositório acadêmico que possibilitasse aos
alunos realizar a publicação de seus artigos, práticas profissionais,
interdisciplinares, TCCs e demais pesquisas realizadas na instituição.
Também foi verificado por meio da aplicação de um questionário, que
mais da metade dos acadêmicos (cerca de 62,1\% das respostas) apresentavam
interesse máximo em uma escala de 1 a 5, por um repositório acadêmico
institucional.

Ao pesquisar sobre repositórios acadêmicos existentes em plataformas
agregadoras como o OpenDOAR\footnote{https://v2.sherpa.ac.uk/opendoar/},
foi constatado que a maioria dos repositórios nacionais utilizam como
base o DSpace, um \emph{software open source} escrito em Java para
criação de repositórios acadêmicos institucionais,
sendo minoria as instituições que utilizam um sistema de repositório
institucional diferente deste.

No trabalho relacionado de \cite{GarciaRodrigoMoreira2019DdnB} é possível
verificar que por se tratar de um \emph{software open source},
fica a critério das instituições as questões pertinentes a personalização
do DSpace, segurança e privacidade dos dados, \emph{backup} dos arquivos,
e a realização das atualizações de versão, o que resulta em um ambiente
fragmentado onde algumas instituições utilizam versões antigas do \emph{software},
que por exemplo não funcionam de forma adequada em dispositivos móveis como
relatado no trabalho relacionado de \cite{FernandesMacedes:2019}.

Possuindo como inspiração os tópicos anteriormente citados,
este trabalho de conclusão de curso possui como objetivo e principal
contribuição cientifica, a utilização de tecnologias baseadas em
nuvem para realizar a análise e desenvolvimento de uma nova plataforma
de repositório acadêmico, que possibilite a interação entre os alunos
e orientadores, englobando o processo de correção e envio de revisões
de publicações dentro da própria plataforma, e que seja uma alternativa
as atuais plataformas de repositórios existentes.

Com a realização desta pesquisa, buscou-se trazer contribuição
tanto para os acadêmicos da faculdade SETREM, quanto para outras
instituições que possam optar pela utilização do repositório
acadêmico desenvolvido. Também houve o proposito de expor a
importância da utilização de repositórios acadêmicos para realizar
a preservação digital dos materiais produzido, evitando assim a perda
ou esquecimento da produção acadêmica e cientifica realizada dentro
da instituição.

Com o aprendizado adquirido durante o desenvolvimento deste trabalho
de conclusão de curso, foi possível realizar o aprofundamento do
conhecimento nas áreas de tecnologia da informação, nuvem computacional,
repositórios acadêmicos, gestão de conhecimento, linguagens de programação
e demais tecnologias utilizadas durante o desenvolvimento da pesquisa.

\section{Problema} \label{sec::Problem}

Como problema, foram ressaltados dois pontos principais que motivaram a
escolha do tema desta pesquisa, sendo o primeiro a necessidade de reunir
as publicações acadêmicas produzidas dentro de instituições de ensino
superior em um repositório institucional, que forneça rápido acesso às
publicações científicas produzidas dentro da instituição, tanto de forma
interna para os estudantes da instituição, quanto de forma externa caso
a instituição opte por disponibilizar o seu acervo de forma pública.

Já o segundo ponto, envolve a forma como ocorre o processo de publicação,
correção e revisão por parte dos orientadores, que em geral envolve a
troca de vários emails entre os orientadores e os orientandos,
processo o qual poderia ser melhorado por meio de um repositório
acadêmico que suporte um esquema de publicação de revisões e correções.

Levando em consideração estes pontos, o problema desta pesquisa
foi definido como: Como projetar um repositório de trabalhos
acadêmicos baseado em serviços de nuvem, que envolva o processo
de correções e envio de revisões das publicações dentro
da própria plataforma?

\section{Hipóteses} \label{sec::Hypothesis}
\begin{enumerate}
      \item O recurso de Full Text Search do banco de dados PostgreSQL
            pode ser utilizado como uma alternativa viável para realizar
            as consultas por publicações dentro do repositório acadêmico
            (menos de 5 segundos por consulta), mesmo em bases de dados
            com mais de 30 mil publicações.

      \item O processo desenvolvido para extração do texto das publicações
            para pesquisa e indexação é rápido o suficiente para não
            necessitar de processamento em segundo plano (menos de 5 segundos),
            mesmo em publicações com mais de 200 páginas.

      \item É possível manter um ambiente em nuvem contendo os serviços
            de backend, frontend, banco de dados e armazenamento das
            publicações em nuvem, com um orçamento máximo de R\$ 500,00
            por mês.

\end{enumerate}


\section{Metodologia} \label{sec:Methodology}

Conforme \citep[p. 15]{LOVATO:metodologia} a metodologia de pesquisa
consiste em um ramo da filosofia da ciência, que estuda os métodos que
os cientistas podem utilizar para chegar nos resultados de seus estudos.
Em outras palavras, tem como objetivo estudar os métodos que visam
conduzir um aumento do conhecimento sobre o tema pesquisado, preocupando-se
com o raciocínio, procedimentos e técnicas que podem ser utilizadas para
dar credibilidade aos resultados.

\subsection{Abordagem}

Os métodos de abordagem segundo \citep[p. 29]{LOVATO:metodologia} podem ser
divididos em dois grupos, o primeiro consiste no tipo de raciocínio que
é utilizado para se chegar aos resultados e conclusões. Já o segundo
se relaciona com a utilização, ou não, de análise numérica e estatística.

Segundo o mesmo autor, a abordagem quantitativa é utilizada quando
as conclusões são obtidas por meio de um conjunto de dados numéricos
e analise estatística, busca compreender melhor situações onde é possível
estabelecer relação entre variáveis, correlações ou causa-efeito.

Já segundo \cite{GIL:metodologia} a abordagem dedutiva, conforme a definição clássica,
é o método em que parte-se de um conceito geral, onde parte dos
conceitos são conhecidos como verdadeiros ou falsos, e se chega
a uma conclusão particular, por meio da aplicação da lógica.

Para o desenvolvimento deste trabalho, foram utilizadas as abordagens
dedutiva e quantitativa, respetivamente a primeira abordagem foi utilizada
de forma a utilizar o raciocínio lógico sobre pesquisas e repositórios
acadêmicos já existes, para melhor compreender o escopo da pesquisa,
e assim propor soluções mais adequadas ao problema.

Já a segunda abordagem foi utilizada para mensurar e analisar
os dados numéricos referentes ao tempo de consultas e arquivamento
das publicações no repositório acadêmico proposto, além dos
custos envolvendo o armazenamento das publicações em nuvem.

\subsection{Procedimentos}

São apresentados neste tópicos os procedimentos utilizados
durante a realização deste trabalho, como a pesquisa bibliográfica e
o método de pesquisa ação.

\subsubsection{Pesquisa Bibliográfica}

Conforme \citep[p. 183]{LAKATOS2003:metodologia}, a pesquisa bibliográfica
abrange toda a bibliografia que se refere ao tema da pesquisa,
incluindo desde publicações avulsas, boletins, jornais, revistas,
livros, monografias, até meios de comunicação via rádio,
gravações e filmes, tendo como finalidade realizar o contanto
entre o pesquisador e o material já existente sobre o tema pesquisado.

Este procedimento foi utilizado por meio da realização de pesquisas
em livros, artigos e revistas, com a finalidade de se obter
conhecimentos sobre o tema pesquisado, e outros sistemas
já existentes com propósitos semelhantes.

\subsubsection{Pesquisa Ação}

Como elicitado por \citep[p. 45]{LOVATO:metodologia}, a pesquisa ação
diferente de outras abordagens em que o principal
embasamento consiste na literatura, o problema é real, e não existe
um plano previamente definido e inflexível, mas sim um refinamento
constante entre planejar, agir avaliar e refletir.

Por se tratar do desenvolvimento de um \emph{software}, o procedimento
de pesquisa ação foi utilizado na forma da pesquisa,
aplicação e avaliação de tecnologias existentes, e que possam
ser utilizadas durante o desenvolvimento do repositório
acadêmico proposto.

\subsection{Validação das Hipóteses}

Durante a elaboração do presente trabalho, foram elencadas
três hipóteses, sendo a  primeira "O recurso de Full Text
Search do banco de dados PostgreSQL pode ser utilizado como
uma alternativa viável para realizar as consultas por publicações
dentro do repositório acadêmico (menos de 5 segundos por consulta),
mesmo em bases de dados com mais de 30 mil publicações".
Esta hipótese pode ser validada por meio da preparação de uma base
de dados de testes, que contenha mais de 30 mil publicações, e
em seguida, realizar uma mesma consulta diversas vezes, realizando
a média de tempo de execução.

A segunda hipótese é apresentada como "O processo desenvolvido
para extração do texto das publicações para pesquisa e indexação
é rápido o suficiente para não necessitar de processamento em
segundo plano (menos de 5 segundos), mesmo em publicações com mais
de 200 páginas". Esta segunda hipótese pode ser validada por meio
de um teste de desempenho, que consiste realizar a publicação de uma
mesma pesquisa com mais de 200 páginas diversas vezes, e mensurar
o tempo médio para conclusão do processo.

Já a terceira hipótese, consiste em "É possível manter um ambiente
em nuvem contendo os serviços de backend, frontend, banco de dados
e armazenamento das publicações em nuvem, com um orçamento máximo
de R\$ 500,00 por mês". Esta hipótese pode ser validada por meio
da comparação dos preços de diferentes provedores de nuvem, com
base nos requisitos mínimos estipulados para o repositório
acadêmicos proposto.

\subsection{Técnicas}

Conforme \citep[p. 174]{LAKATOS2003:metodologia} as técnicas podem
ser definidas como um conjunto de procedimentos que servem a
uma ciência ou arte, sendo utilizados para trazer tais
conceitos a parte prática, tendo em mente que toda ciência utiliza
de incontáveis técnicas para a obtenção de seus resultados.

Durante o desenvolvimento desta pesquisa, por se tratar do
desenvolvimento de um \emph{software}, foi utilizado da técnica
de prototipação de telas por meio de \emph{mockups} realizados
com a ferramenta Figma.

Em tradução livre de \cite{uzayr:mockups} um \emph{software mockup}
consiste em um desenho de uma página ou aplicação web,
que é desenvolvida para trazer vida a uma ideia, permitindo
que um designer possa examinar como diferentes elementos visuais
trabalham juntos. Os \emph{mockups} também permitem que as partes
interessadas ou \emph{Stakeholders} do projeto possam verificar como a
interface irá parecer, enquanto sugerem mudanças
apropriadas em relação a cores, imagens e estilos.

Também foram utilizadas as técnica de modelagem ER (Entidade Relacionamento)
por meio da ferramenta pgModeler para o desenvolvimento do banco de dados,
e testes para verificar o funcionamento do \emph{software} desenvolvido.


\section{Orçamento} \label{sec:budget}

De acordo com \citep[p. 128]{LAKATOS2021:metodologia} o orçamento
tem como objetivo responder a questão "quanto será necessário
despender?", distribuindo os gastos com o projeto em vários items.

\begin{table}[H]
      \caption{Orçamento}
      \begin{tabular}{|p{6.4cm}|c|r|r|}
            \hline
            {\textbf{Item}}                         & {\textbf{Quantidade}} & {\textbf{Valor Unitário}} & {\textbf{Valor Total}} \\ \hline
            {{Impressões}}                          & {1000}                & {R\$ 0,30 }               & {R\$ 300,00}           \\ \hline
            {{Capa}}                                & {2}                   & {R\$ 25,00 }              & {R\$ 50,00}            \\ \hline
            {{Espirais}}                            & {3}                   & {R\$ 5,00 }               & {R\$ 15,00}            \\ \hline
            {{Serviços de Designer}}                & {1}                   & {R\$ 350,00 }             & {R\$ 350,00}           \\ \hline
            {{Serviços de Banco de Dados em nuvem}} & {1}                   & {R\$ 300,00 }             & {R\$ 300,00}           \\ \hline
            {{Serviços de hospedagem em nuvem}}     & {2}                   & {R\$ 100,00 }             & {R\$ 200,00}           \\ \hline
            {{Serviços de armazenamento em nuvem}}  & {1}                   & {R\$ 30,00 }              & {R\$ 30,00}            \\ \hline
      \end{tabular}
\end{table}

\section{Cronograma de Atividades} \label{sec:schedule_activities_table}

Em conformidade com \citep[p. 128]{LAKATOS2021:metodologia} o cronograma
visa responder a pergunta “quando?”, sendo elaborado de forma a
representar a previsão do tempo necessário para realização da pesquisa.

\begin{table}[H]
      \caption{Cronograma}
      \begin{tabular}{|p{7cm}|l|l|l|l|l|l|l|}
            \hline
            \multicolumn{1}{|c|}{}                                      & \multicolumn{7}{c|}{\textbf{2022}}                                                                                                                                                                   \\ \cline{2-8}
            \multicolumn{1}{|c|}{\multirow{-2}{*}{\textbf{Atividades}}} & \textbf{Mai}                       & \textbf{Jun}             & \textbf{Jul}             & \textbf{Ago}             & \textbf{Set}             & \textbf{Out}             & \textbf{Nov}             \\ \hline
            Desenvolvimento do projeto                                  & \cellcolor[HTML]{000000}           & \cellcolor[HTML]{000000} &                          &                          &                          &                          &                          \\ \hline
            Entrega e validação do projeto                              &                                    & \cellcolor[HTML]{000000} & \cellcolor[HTML]{C0C0C0} &                          &                          &                          &                          \\ \hline
            Organização de Dataset de trabalhos acadêmicos              &                                    &                          & \cellcolor[HTML]{C0C0C0} &                          &                          &                          &                          \\ \hline
            Desenvolvimento de protótipos                               &                                    &                          & \cellcolor[HTML]{C0C0C0} & \cellcolor[HTML]{C0C0C0} &                          &                          &                          \\ \hline
            Desenvolvimento do repositório acadêmico                    &                                    &                          & \cellcolor[HTML]{C0C0C0} & \cellcolor[HTML]{C0C0C0} & \cellcolor[HTML]{C0C0C0} & \cellcolor[HTML]{C0C0C0} &                          \\ \hline
            Realização de testes no repositório acadêmico               &                                    &                          &                          & \cellcolor[HTML]{C0C0C0} & \cellcolor[HTML]{C0C0C0} & \cellcolor[HTML]{C0C0C0} & \cellcolor[HTML]{C0C0C0} \\ \hline
            Documentação dos resultados                                 &                                    &                          &                          & \cellcolor[HTML]{C0C0C0} & \cellcolor[HTML]{C0C0C0} & \cellcolor[HTML]{C0C0C0} & \cellcolor[HTML]{C0C0C0} \\ \hline
            Entrega do Relatório Final                                  &                                    &                          &                          &                          &                          &                          & \cellcolor[HTML]{C0C0C0} \\ \hline
            Apresentação do Relatório Final                             &                                    &                          &                          &                          &                          &                          & \cellcolor[HTML]{C0C0C0} \\ \hline
      \end{tabular}
\end{table}

\begin{table}[H]
      \begin{tabular}{|
                  >{\columncolor[HTML]{C0C0C0}}l |l|}
            \hline
                                                            & Proposto  \\ \hline
            \cellcolor[HTML]{000000}{\color[HTML]{000000} } & Realizado \\ \hline
      \end{tabular}
\end{table}
