\chapter{Plano de Estudo e Pesquisa} \label{chap:ResearchPlan}




\section{Tema} \label{sec::Theme}
Análise e desenvolvimento de um Repositório Institucional de Trabalhos Acadêmicos.


\subsection{Delimitação do Tema} \label{subsec::ThemeDelimitation}

A delimitação do tema se dará como a Análise e desenvolvimento de um Repositório Institucional de Trabalhos Acadêmicos, tendo como objetivo reunir e preservar as publicações acadêmicas e científicas produzidas em âmbito universitário, além de unificar o processo de publicações e correções por parte dos orientadores em uma única plataforma.

O desenvolvimento deste projeto de pesquisa foi realizado durante o período de maio a julho de 2022, pelo acadêmico Lucas Machado Alf do curso bacharelado em Sistema de Informação na Sociedade Educacional Três de Maio – SETREM.

\section{Objetivo Geral} \label{sec:objective}

Desenvolver um repositório institucional de trabalhos acadêmicos que envolva o processo de correções e envio de revisões de publicações dentro da própria plataforma.

\subsection{Objetivos Específicos}
\begin{enumerate}
    \item Explorar ferramentas existentes de repositório acadêmico.
    \item Pesquisar sobre trabalhos e artigos relacionados.
    \item Pesquisar sobre formas de armazenamento de documentos em nuvem.
    \item Investigar mecanismos de busca e análise textual de documentos. (Ex. ElasticSearch e Full Text Search).
    \item Elicitar os requisitos para o desenvolvimento do repositório acadêmico.
    \item Definir as tecnologias que serão utilizadas para o desenvolvimento (Ex: banco de dados, linguagens e frameworks).
    \item Desenvolver uma aplicação web de repositório acadêmico.
    \item Avaliar a usabilidade do repositório acadêmico desenvolvido.
    \item Realizar testes no repositório acadêmico desenvolvido, verificando o desempenho da aplicação em relação ao crescimento da quantidade de registros publicados na plataforma.

\end{enumerate}


\section{Justificativa}\label{sec:justification}

O tema desta pesquisa surgiu como uma continuação direta a um projeto
de Prática Profissional desenvolvido pelo próprio autor, durante o sétimo
semestre da graduação de Sistemas de Informação, na Sociedade Educacional
Três de Maio – SETREM, que tinha como objetivo analisar o atual processo
de armazenamento e acesso as publicações produzidas dentro da faculdade,
e realizar a representação gráfica dos processos por meio de diagramas BPMN
(\emph{Business Process Model and Notation}).

Durante o desenvolvimento da pesquisa, foi constatado que a faculdade
não possuía um sistema de repositório acadêmico que permita aos
alunos realizar a publicação de seus artigos, práticas profissionais,
interdisciplinares, TCCs e demais pesquisas realizadas na instituição.
Também foi verificado por meio da aplicação de um questionário, que
mais da metade dos acadêmicos (cerca de 63,1\% das respostas) apresentavam
interesse máximo em uma escala de 1 a 5, por um repositório acadêmico
institucional.

Ao pesquisar sobre repositórios acadêmicos existentes em plataformas
agregadoras como o OpenDOAR, foi constatado que a maioria dos repositórios
nacionais utilizam como base o DSpace, um \emph{software open source}
escrito em Java para criação de repositórios acadêmicos institucionais,
sendo minoria as instituições que utilizam um sistema de repositório
institucional diferente deste.

Também foi verificado que por se tratar de um \emph{software open source},
fica a critério das instituições as questões referentes a segurança e
privacidade dos dados, \emph{backup} dos arquivos e realizar as atualização
das versões do DSpace, o que resulta em um ambiente fragmentado onde algumas
instituições utilizam versões antigas do software, que por exemplo não
funcionam de forma adequada em dispositivos móveis como relatado no
trabalho relacionado de \cite{FernandesMacedes:2018}.

Possuindo como inspiração os tópicos anteriormente citados,
este projeto de pesquisa possui como objetivo, e principal contribuição
cientifica, a utilização de tecnologias atualmente populares no mercado,
como recursos de armazenamento em nuvem, e ferramentas como o ReactJS e
.NET Core, para o desenvolvimento de uma nova plataforma de repositório
acadêmico, que possibilite a interação entre os alunos e
orientadores, englobando o processo de correção e envio de revisões de
publicações dentro da própria plataforma, e que seja uma alternativa
as atuais plataformas de repositórios existentes como o DSpace.

\section{Problema} \label{sec::Problem}

Como problema, foram ressaltados dois pontos principais que motivaram a escolha do tema desta pesquisa, sendo o primeiro a necessidade de reunir as publicações acadêmicas produzidas dentro de instituições de ensino superior em um repositório institucional, que fornece rápido acesso às publicações científicas produzidas dentro da instituição, tanto de forma interna para os estudantes da instituição, quanto de forma externa caso a instituição opte por disponibilizar o seu acervo de forma pública.

Já o segundo ponto, envolve a forma como ocorre o processo de publicação, correção e revisão por parte dos orientadores, que em geral envolve a troca de vários emails entre os orientadores e os orientandos, processo o qual poderia ser melhorado por meio de um repositório acadêmico que suporte um esquema de publicação de revisões e correções.

Levando em consideração estes pontos, o problema desta pesquisa foi definido como: Como projetar um repositório de trabalhos acadêmicos, que envolva o processo de correções e envio de revisões das publicações dentro da própria plataforma?


\section{Hipóteses} \label{sec::Hypothesis}
\begin{enumerate}
    \item O recurso de Full Text Search do banco de dados PostgreSQL pode ser utilizado como uma alternativa viável para realizar as consultas por publicações dentro do repositório acadêmico (menos de 3 segundos por consulta).
    \item O processo desenvolvido para extração do texto das publicações para pesquisa e indexação é rápido o suficiente para não necessitar de processamento em segundo plano (menos de 5 segundos), mesmo em relatórios com mais de 200 páginas.
\end{enumerate}


\section{Metodologia} \label{sec:Methodology}

Conforme \citep[p. 15]{LOVATO:metodologia} a metodologia de pesquisa
consiste em um ramo da filosofia da ciência, que estuda os métodos que
o cientista pode utilizar para chegar nos resultados de seus estudos.
Em outras palavras, tem como objetivo estudar os métodos que visam
conduzir um aumento do conhecimento sobre o tema pesquisado, preocupando-se
com o raciocínio, procedimentos e técnicas que podem ser utilizadas para
dar credibilidade aos resultados.

\subsection{Abordagem}

Os métodos de abordagem segundo \citep[p. 29]{LOVATO:metodologia} podem ser
divididos em dois grupos, o primeiro consiste no tipo de raciocínio que
é utilizado para se chegar aos resultados e conclusões. Já o segundo
se relaciona com a utilização, ou não, de analise numérica e estatística.

Para o desenvolvimento deste trabalho, foram utilizadas as abordagens
quantitativas e de pesquisa-ação, onde a abordagem quantidade foi utilizada
para mensurar e avaliar o tempo de processamento para a extração de texto
das publicações

\subsection{Procedimentos}



\subsection{Técnicas}

\subsection{Validação das Hipóteses}


\section{Orçamento} \label{sec:budget}

%table example

\section{Cronograma de Atividades} \label{sec:schedule_activities_table}

%table example