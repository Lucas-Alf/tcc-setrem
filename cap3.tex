% Chapter 3
\chapter{Resultados}\label{chap:Results}

Neste capítulo são apresentados os resultados do trabalho desenvolvido,
apresentando a criação do repositório institucional proposto, desde a
sua concepção, \emph{backlog} do produto, protótipos, resultado final
e validação das hipóteses.

\section{Concepção do Repositório Institucional}

A concepção do repositório institucional proposto foi realizada com base
nos desafios e problemas relatados nos trabalhos relacionados, e na análise
dos pontos positivos e negativos encontrados em outros sistemas de repositório
institucional existentes.

Como um dos principais diferenciais do repositório proposto, é que desde o
inicio de seu desenvolvimento ele foi projetado para ser facilmente implantado
e executado em ambientes baseados em nuvem, sendo assim todas as suas configurações
podem ser realizadas por meio de variáveis de ambiente, que em geral podem ser
facilmente definidas pelo próprio painel ou \emph{dashboard} fornecido pelo
provedor de nuvem. A ideia é nunca precisar acessar remotamente a máquina
em que a aplicação está sendo executada para realizar qualquer tipo de configuração.

Outro ponto de diferença é que o repositório proposto somente aceita arquivos no
formato PDF, visando mitigar problemas relacionados a formatação do documento, e
buscando garantir que o usuário ao acessar as publicações, visualize
o documento da mesma forma que o autor originalmente a escreveu. Esta redução
do escopo de formatos de arquivos aceitos pelo repositório, também garante que
todas as publicações estejam normalizadas em um mesmo formato, permitindo a
utilização de técnicas mais especializadas para a extração de texto dos documentos.

Para realizar a escolha do nome e identidade visual do repositório, foram buscados
por termos que remetessem a "Repositório", "Acesso Aberto'' e "Pesquisa", chegando
a escolha do nome RESOAR, uma sigla em inglês para \emph{Research Open Access Repository}.
A sigla "OAR'' que remete a \emph{Open Access Repository} já é baste difundida, e
pode ser encontrada em nomes como o OpenDOAR\footnote{https://v2.sherpa.ac.uk/opendoar/} e
ROARMAP\footnote{https://roarmap.eprints.org/}.

\begin{figure}[H]
    \caption{Identidade visual do repositório}
    \centering
    \frame{\includegraphics[scale=0.0558]{img/resoar.png}}
    \label{fig:resoar}
    \source{Do próprio autor}
\end{figure}

A Figura \ref{fig:resoar} apresenta a logo ou identidade visual do repositório,
que foi elaborada por um designer contratado da região de Três de Maio - RS,
que desenvolveu tanto a identidade visual quanto os primeiros protótipos do repositório.

Em sua composição é possível perceber um simbolo de lupa embutido na primeira letra "R''
representando a pesquisa. A cor padrão da logo foi escolhida como o laranja com
o código HEX \#ff5400, porem o simbolo também pode ser exibido em outras variações
em preto e branco, escala de cinza ou outras cores.


\section{Backlog do produto} \label{sec:dev}

\section{Protótipos} \label{sec:exp}



