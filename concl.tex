\chapter*{Conclusão} \label{chap:concl}

Cada vez mais os repositórios institucionais estão sendo adotados por instituições de
ensino superior, com o objetivo de reunir, preservar e oferecer o fácil acesso ao material
científico produzido ao decorrer dos anos. Sendo por meio deste movimento de preservação
digital da produção científica, que iniciativas de acesso aberto se tornam possíveis,
possibilitando que uma produção científica escrita hoje possa ser vista e acessada
de forma gratuita pelo mundo todo, podendo ser utilizada como base ou inspiração para
diversas outras pesquisas.

Também torna-se relevante a existência do desenvolvimento ativo de diferentes
alternativas de \emph{softwares} utilizados para elaborar tais repositórios
institucionais, tendo em vista o constante progresso tecnológico, e a necessidade
de que estas ferramentas não fiquem ultrapassadas ou monopolizadas por uma única
opção de \emph{software}. Além da necessidade da existência de padrões, para que
estas alternativas não criem um ambiente fragmentado de sistemas e padrões que não
são compatíveis entre si.

O objetivo deste trabalho de desenvolver um repositório institucional de pesquisas
acadêmicas baseado em nuvem, visando reunir e preservar as publicações acadêmicas
e científicas produzidas em âmbito universitário, além de unificar o processo
de publicações e correções por parte dos orientadores em uma única plataforma,
foi atingido, por meio do desenvolvimento de um novo \emph{software web} de
repositório institutional que foi denominado de \emph{RESOAR}
(\emph{Research Open Access Repository}).

O \emph{RESOAR} foi desenvolvido tendo como objetivo utilizar de tecnologias baseadas
em nuvem para realizar o armazenamento das publicações acadêmicas, possuindo suporte
a serviços de \emph{Object Storage} como o \emph{Amazon S3} e \emph{Digital Ocean Spaces}.

Além disto, todas as configurações tanto do \emph{backend} quanto do \emph{frontend}
podem ser realizadas por meio de variáveis de ambiente, visando facilitar a implantação
do sistema em ambientes baseados em nuvem por meio de imagens \emph{docker}, tendo como
principio a ideia de nunca precisar acessar remotamente o container para realizar qualquer
tipo de configuração.

A primeira hipótese deste trabalho apresenta que "O recurso de \emph{Full Text Search} presente
no banco de dados PostgreSQL pode ser utilizado como uma alternativa viável para realizar
as consultas por publicações dentro do repositório acadêmico proposto, (menos de 1 segundo
por consulta), mesmo em bases de dados com mais de 3.500 publicações".

Esta primeira hipótese foi validada por meio da execução de testes de carga e estresse
sobre sobre o \emph{endpoint} de consulta por publicações do \emph{backend} desenvolvido.
Com os dados coletados, foi constatado que no cenário descrito pela hipótese, uma consulta
no repositório institucional possui um tempo médio de 114,54 milissegundos.

Já a segunda hipótese deste trabalho afirma que "O processo desenvolvido para extração do
texto das publicações durante o auto arquivamento é rápido o suficiente para não necessitar
de processamento em segundo plano (menos de 3 segundos), mesmo em publicações com mais de
12.000 palavras, cerca de 40 páginas de texto em português com fonte tamanho 12,
considerando que cada página tenha 300 palavras."

Esta segunda hipótese também foi validada por meio da aplicação de um teste de carga
e estresse sobre o \emph{backend} desenvolvido para o repositório institucional. Por meio
das métricas coletadas, foi constatado que uma publicação com as características
descritas pela hipótese, demora em média 2,30 segundos para ser concluída.

Em suma, o foco deste trabalho de desenvolver um novo repositório institucional
de pesquisas acadêmicas, utilizando de tecnologias baseadas em nuvem, e unificar
o processo de publicações e correções por parte dos alunos e orientadores, foi atingido.
Proporcionando aos envolvidos um grande aprendizado nas diversas áreas que envolvem
o desenvolvimento de \emph{software}, além de possibilitar a aplicação em prática dos
conhecimentos adquiridos ao decorrer da graduação.

Também fica como contribuição, a possibilidade da utilização do \emph{software}
desenvolvido, tanto pela SETREM, quanto a outras instituições que venham desenvolver
interesse pela utilização, ou pela contribuição com o projeto por meio de sugestões
ou melhorias ao \emph{software} de repositório institucional desenvolvido.

Como proposta futura, fica a sugestão da criação de um recurso dentro do próprio
repositório institucional, que permita realizar anotações em cima das publicações
acadêmicas, como a adição de comentários ou rabiscos de caneta. Este recurso poderia
ser especialmente útil para os orientadores, que poderiam adicionar sugestões ou
correções as pesquisas dentro da própria plataforma. No momento em que este trabalho
foi desenvolvido, foram encontradas poucas bibliotecas que permitem este tipo de
edição diretamente pelo o navegador do usuário, sendo em maioria pagas,
inviabilizando a aplicação do recurso no sistema.